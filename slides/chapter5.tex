\input{snippet/beamer_head}
\input{snippet/class_head}

%\includeonlyframes{current}

\title[医学]{第五章\quad 疯狂的实验——医学}
\author[Yixf]{伊现富(Yi Xianfu)}
\institute[TIJMU]{天津医科大学(TIJMU)\\ 生物医学工程与技术学院}
\date{2018年4月}

\input{snippet/beamer_toc.tex}

\begin{frame}
  \frametitle{医学 | 小测验}
  \begin{figure}
    \centering
    \includegraphics[width=0.6\textwidth]{c5.test.01.jpg}
  \end{figure}
\end{frame}

% 2-9,1747
\section{坏血病}
\begin{frame}
  \frametitle{坏血病 | 简介}
  \begin{block}{坏血病}
    \begin{itemize}
      \item 起源:一旦航行时间稍长,大多数船员就会患病。
      \item 症状:牙龈溃烂、关节疼痛、皮下自发性出血等。
      \item 危害:热带病、船难事故和海上作战加在一起都没有它严重。
    \end{itemize}
  \end{block}
  \pause
  \begin{block}{病因猜测}
    \begin{itemize}
      \item 船上空气质量差、有老鼠、肝脏感染病毒、食物加盐过多、天气太热/冷……
      \item 针对各种诱因采取不同的应对手段,然并卵……不能证明其中任何一种手段真的有效。
    \end{itemize}
  \end{block}
\end{frame}

\begin{frame}
  \frametitle{坏血病 | 实验}
  \begin{block}{实验}
    詹姆斯 \textbullet 林德想到要对某些手段进行分类检测:
    \begin{itemize}
      \item 6组,每组2人,隔离单间,治疗2周,正常饮食外加:
      \item 其中4组:饮用苹果酒、醋、稀释的硫酸以及海水
      \item 第5组:由大蒜、芥籽、秘鲁香膏、香胶木树脂混合而成的当时常用的药剂
      \item 第6组:2个橙子和1个柠檬
    \end{itemize}
  \end{block}
  \vspace{-0.5em}
  \pause
  \begin{block}{结果}
    \begin{itemize}
      \item 第6组的2位病人基本恢复健康
      \item 苹果酒显示出了微弱的效果;其他方法几乎毫无用处
    \end{itemize}
  \end{block}
  \vspace{-0.5em}
  \pause
  \begin{block}{结论}
    橙子和柠檬是治疗环血病最有效的手段。
  \end{block}
\end{frame}

\begin{frame}
  \frametitle{坏血病 | 启示}
  \begin{block}{启示}
  \begin{itemize}
    \item 坏血病是一种营养缺乏症。所有症状都由同一个问题引起:缺少黏合身体的“胶水”——骨胶原。人体如果缺乏维生素C,就不能正常合成骨胶原。
    \item 不能自行产生维生素C的生物并不太多,只有蝙蝠、某些猿猴类动物、人类以及豚鼠(误打误撞选用了豚鼠做实验)。
    \item 不要随便推广那些纯粹从理论出发得来的结论,而要在实践中检验一下。
    \item \alert{理论只有经过证明,才具备可信性。} 通过实践来检验效果的方法为现代药物研究活动做出了良好表率。
    \item \alert{为了排除各种不必要的干扰,人们应尽可能将被试者分成相似的小组,再对他们实施不同的治疗。}
    \item \alert{药物测试必须在“双盲”的前提下进行}:病人和医生只有在研究结束后才能知道他们使用的是哪种药物。(防止人们由于事先了解某种药物而产生妨碍实验准确性的心理暗示。)
  \end{itemize}
  \end{block}
\end{frame}

\begin{frame}
  \frametitle{坏血病 | 双盲}
  \begin{figure}
    \centering
    \includegraphics[width=0.8\textwidth]{c4.blind.01.jpg}
  \end{figure}
\end{frame}

\section{自体实验}
\begin{frame}
  \frametitle{自体实验 | 引言}
  \begin{block}{科学怪人 vs. 真英雄}
    有2类实验是人们很难做到的,其一是研究者因其实验一生都得到同事的赞誉,其二是实验令研究者永远被视为可笑的怪人。科学上的真英雄,要在第二类中寻找。
  \end{block}
  \begin{figure}
    \centering
    \includegraphics[width=0.26\textwidth]{c5.hero.01.jpg}
    \includegraphics[width=0.4\textwidth]{c5.hero.02.jpg}
    \includegraphics[width=0.278\textwidth]{c5.hero.03.jpg}
  \end{figure}
\end{frame}

% 1-19,1802
\begin{frame}
  \frametitle{自体实验 | 黄热病}
  \begin{block}{疑问}
    与黄热病人的接触有没有危险?(黄热病能不能在人与人之间传播?)
  \end{block}
  \pause
  \begin{block}{实验}
    \begin{itemize}
      \item 1802年,斯塔宾斯\textbullet 弗斯(18岁),博士论文
      \item 狗/猫,黄热病人的呕吐物:喂食浸泡过的面包(健康),敷在伤口上(健康),注射到颈部静脉(死;注射水也死)
      \item 他自己,呕吐物:敷在伤口上(20几个部位的重复),滴入眼睛,吸入蒸汽,吞下(制成的药片、稀释物)
      \item 他自己,其他体液或身体排泄物(血液、唾液、汗液、尿液)
    \end{itemize}
  \end{block}
  \pause
  \begin{block}{影响}
英雄之举影响甚微。实验主要揭示了黄热病无法通过一些方式传染,而人们想知道的却是:黄热病是怎么传播的?(蚊子,血液!)
  \end{block}
\end{frame}

% 1-197,1968
\begin{frame}
  \frametitle{自体实验 | 螨虫}
  \begin{block}{现象与疑问}
    \begin{itemize}
      \item 给猫治疗耳螨时,女房东和她女儿抱怨很痒。
      \item 耳螨能够传播给人类吗?
    \end{itemize}
  \end{block}
  \vspace{-0.5em}
  \pause
  \begin{block}{实验}
    \begin{itemize}
      \item 耳螨:取自猫耳,显微镜检查以确认是耳螨
      \item 实验:将大约1克混有耳螨的猫的耳垢放到被试(洛佩兹自己)的左耳里(先后三次实验)
    \end{itemize}
  \end{block}
  \vspace{-0.5em}
  \pause
  \begin{block}{结果}
    详细记录:声响、瘙痒、疼痛、耳垢……
  \end{block}
  \vspace{-0.5em}
  \pause
  \begin{block}{启示}
    \begin{itemize}
      \item \alert{如果一项实验无法重复,实验的结果就算不上证据确凿。}
      \item 1994年凭借其论文获得搞笑诺贝尔奖!
    \end{itemize}
  \end{block}
\end{frame}

% 1-278,1984
\begin{frame}
  \frametitle{自体实验 | 胃溃疡}
  \begin{block}{已知}
    \begin{itemize}
      \item 沃伦在针对胃黏膜炎患者的细胞实验中发现了一种不知名的细菌。
      \item 早在十九世纪就有研究者在人类和动物的胃中发现了螺旋杆菌。
    \end{itemize}
  \end{block}
  \pause
  \begin{block}{疑问}
    这些细菌与炎症有关系吗?
  \end{block}
  \pause
  \begin{block}{实验}
    马歇尔为一位病人施用抗菌素,结果那种不知名的病菌和胃炎同时消失了。
  \end{block}
  \pause
  \begin{block}{结论}
    细菌引发胃炎! vs. 细菌产生后,在胃的伤口处繁殖。
  \end{block}
\end{frame}

\begin{frame}
  \frametitle{自体实验 | 胃溃疡}
  \begin{block}{科赫法则}
    科赫在1882年就证明病菌的致病性提出了4项要求:
    \begin{enumerate}
      \item 细菌必须能够在每一个病例中找到
      \item 细菌必须能够在体外培养
      \item 培养的细菌能够使实验动物染病
      \item 细菌必须能再次从实验动物体内获得并培养
    \end{enumerate}
  \end{block}
  \vspace{-0.5em}
  \pause
  \begin{block}{科赫法则一}
    沃伦和马歇尔在病人的胃壁上总能找到这种细菌。
  \end{block}
  \vspace{-0.5em}
  \pause
  \begin{block}{科赫法则二}
    \begin{itemize}
      \item 一般来说,细菌在培养盘中繁殖的时间不会超过2天。
      \item 细菌培养48小时,一点儿增长的迹象都没有(大约30次无果的实验) $\Longrightarrow$ (一次意外)培养皿在暖箱中放置5天,细菌数量奇迹般地增多了。
    \end{itemize}
  \end{block}
\end{frame}

\begin{frame}
  \frametitle{自体实验 | 胃溃疡}
  \begin{block}{科赫法则三、四}
    \begin{itemize}
      \item 给2只老鼠、2只小猪注射细菌,都没有染病的症状。
      \item 感染无法在动物身上得到证明,那就自己充当实验动物。
    \end{itemize}
  \end{block}
  \pause
  \begin{block}{自体实验}
    \begin{enumerate}
      \item 取样确认用于自身实验的胃是健康的
      \item 喝下收集自病人的细菌培养液
      \item 第十天,在胃出口取2份实验试样(一份检验,一份分离细菌进行培养)
    \end{enumerate}
  \end{block}
  \pause
  \begin{block}{结果}
    \begin{itemize}
      \item 马歇尔感染了胃炎。
      \item 培养液和10天前喝下去的是一样的。
    \end{itemize}
  \end{block}
\end{frame}

\begin{frame}
  \frametitle{自体实验 | 胃溃疡}
  \begin{block}{启示}
    \begin{itemize}
      \item 细菌(幽门螺旋杆菌)可以引发胃炎。
      \item 马歇尔在2005年获得诺贝尔奖。
      \item 他的实验掀起了用感染来解释其他疾病的新的热潮。
      \item 时至今日,借助细菌和病毒的手段,研究者对医治精神分裂症和心肌梗、风湿和糖尿病有了新的希望。迄今这些猜测只有少数得到了证实。
      \item \alert{人体“第二基因组”——肠道菌群}。
    \end{itemize}
  \end{block}
\end{frame}

% 2-35,1882
% 2-44,1905
\begin{frame}
  \frametitle{自体实验 | “自缢”}
  \begin{block}{疑问}
    \begin{itemize}
      \item 哪种绞刑方式最快致死?“升挂”还是“甩坠”?
      \item 人在缢亡的时候会有什么感觉?
    \end{itemize}
  \end{block}
  \vspace{-0.3em}
  \pause
  \begin{block}{实验}
    \begin{itemize}
      \item 数据统计:172名自杀者,分类,地点和季节,工具,怎么打结……
      \item 自体实验:米诺维奇,指压颈动脉,“不完全缢亡”,把自己吊起来
      \item 发表论文:《缢亡研究》,238页,给出了“缢亡”的严谨科学定义
    \end{itemize}
  \end{block}
  \vspace{-0.3em}
  \pause
  \begin{block}{结果}
    \begin{itemize}
      \item 颈部伤害多种多样,喉骨和舌骨的骨折几乎不可避免。
      \item 绳索在脖子上的位置起着决定性的作用。
      \item 脑部供血中断才是导致死亡的主要原因。
      \item 米诺维奇的实验是法医学的经典实验。
    \end{itemize}
  \end{block}
\end{frame}

% 2-94,1954
\begin{frame}
  \frametitle{自体实验 | 撞击}
  \begin{block}{疑问}
    \begin{itemize}
      \item 飞机以超音速飞行时,飞行员突然离开飞机时会有什么感觉?
      \item 强大的气流会冲击他们的身体,使他们瞬间减速,这么巨大的力量会不会致人死亡?
    \end{itemize}
  \end{block}
  \pause
  \begin{block}{实验与结果}
    \begin{itemize}
      \item 实验:约翰\textbullet 保罗\textbullet 斯塔普,火箭滑橇,数十次
      \item 结果:淤青,多次骨折,险些失明,《机械力对活体组织造成的影响》
    \end{itemize}
  \end{block}
  \pause
  \begin{block}{启示}
    \begin{itemize}
      \item 人体可以承受40倍以上的重力加速度。
      \item 在实验的启发下,设计师改良了飞行员座椅和安全绑带。
      \item 斯塔普率先大力推行汽车防撞安全带。
    \end{itemize}
  \end{block}
\end{frame}

% 1-121,1946
\section{感冒}
\begin{frame}
  \frametitle{感冒 | 实验准备}
  \begin{block}{观点与现象}
    \begin{itemize}
      \item 大众观点:寒冷本身引发感冒。
      \item 去往极地进行长期考察的研究人员并不感冒。
      \item 爱斯基摩人在最冷的冬天不会生病,却在有外来船只驶入港口时生病。
    \end{itemize}
  \end{block}
  \pause
  \begin{block}{实验准备:在实验之前排除一切由外部带入的感冒可能}
    \begin{itemize}
      \item 2人一组住进套房,接下来的10天里除室友以外,要与一切没有保护装置的人保持10米以上的距离
      \item 允许散步,但不得进入医院大楼和使用交通工具
      \item 医生和护士进行检查时身穿防护服、脸配面具
      \item 每日3餐被装入保温容器,按时送到套房门前
    \end{itemize}
  \end{block}
\end{frame}

\begin{frame}
  \frametitle{感冒 | 实验}
  \begin{block}{实验}
    \begin{itemize}
      \item 将被试分成3组,每组6人
      \item 第一组接受鼻腔注入(将一位感冒患者的鼻腔分泌物过滤并稀释后滴入鼻中)
      \item 第二组接受冷却处理(先热水浸泡,后在通风的走廊里坚持待上半个小时,其余时间必须穿着湿袜子)
      \item 第三组同时施用冷却处理和鼻腔注入两种办法
    \end{itemize}
  \end{block}
  \pause
  \begin{block}{结果}
    \begin{itemize}
      \item 既受病毒感染又遭降温处理的人中有4人生病
      \item 只受病毒感染的人中有2人生病
      \item 单纯受凉并未造成任何人感冒
    \end{itemize}
  \end{block}
\end{frame}

\begin{frame}
  \frametitle{感冒 | 重复实验}
  \begin{block}{重复}
      20世纪50、60年代,上百名被试,一系列实验
  \end{block}
  \pause
  \begin{block}{结果}
    \begin{itemize}
      \item 一致:没有显示感冒与之前的受寒有何关联
      \item 不一致:只受病毒感染的小组中的感冒人数比既受感染又遭寒冷的小组中的感冒人数翻了一番
    \end{itemize}
  \end{block}
  \pause
  \begin{block}{启示}
    \begin{itemize}
      \item \alert{科学在面对大众观念时处境艰难!}
      \item 为什么感冒的冬季发病率高于夏季?(至今尚不十分明确)
      \item 冬季感冒大量发病与虚弱的免疫系统及房间的干燥空气并无关系。
      \item 更可能的情况是:冬季房间通风不良,冬季太阳照射相对较弱(紫外线可以杀灭病原)。
    \end{itemize}
  \end{block}
\end{frame}

\section{史海撷华}
% 1-4,1600
\begin{frame}
  \frametitle{史海撷华 | 称盘上的生活}
  \begin{block}{实验}
    \begin{itemize}
      \item 圣多里奥(\alert{量化实验医学}的鼻祖),《静态医学医疗术》
      \item 在圣多里奥的房子里,一切都悬于称上:床、工作台、椅子……
      \item 在称盘上度过的时间之长(30年)无人能及
      \item 记录自身体重的点滴变化:进食的食物重量,排泄的废物重量,……
    \end{itemize}
  \end{block}
  \pause
  \begin{block}{论断}
    \begin{itemize}
      \item 人们所排泄的大小便仅占所进食的食品重量的很小一部分(排汗,第一个测量这一重量的人)
    \end{itemize}
  \end{block}
\end{frame}

% 2-27,1875
\begin{frame}
  \frametitle{史海撷华 | 视动鼓}
  \begin{block}{视动鼓}
    \begin{itemize}
      \item “条纹筒”,“视动鼓”,“视动性眼震仪”。
      \item 研究恶心呕吐现象的理想工具。
      \item 检测晕车药的药效。
    \end{itemize}
  \end{block}
  \pause
  \begin{block}{结果}
    \begin{itemize}
      \item 被试者会反复出现瞬时性的错觉(以为转动的不是圆筒而是自己)。
      \item 亚洲人出现恶心的症状明显要比欧洲人快。
      \item 恶心和呕吐是由不同进程引起的。
    \end{itemize}
  \end{block}
  \pause
  \begin{block}{未解之谜}
    \begin{itemize}
      \item 视动鼓中的被试者究竟为何感到不适?
      \item 相互矛盾的感官信号为什么一定会导致恶心呢?
    \end{itemize}
  \end{block}
\end{frame}

\begin{frame}
  \frametitle{史海撷华 | 其他}
  \begin{block}{真正的疯狂}
    \begin{itemize}
      % 1-37,1889
    \item 返老还童——查尔斯 \textbullet 布朗,注射过滤后的浆液(捣碎的睾丸+蒸馏水),安慰剂效应、\alert{荷尔蒙疗法}的先驱
      % 1-47,1899
      \item 脊椎麻醉——医生助手齐上阵(日后反目),频繁详细的实验记录
      % 1-93,1928
      \item 注射曼巴蛇毒——眼镜蛇 vs. 曼巴蛇,错误的估计、巨大的代价
      % 1-149,1955
      \item “白衣行动”——153项生物武器实验
    \end{itemize}
  \end{block}
\end{frame}


\input{snippet/class_tail}
