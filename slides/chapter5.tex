\input{snippet/beamer_head}
\input{snippet/class_head}

%\includeonlyframes{current}

\title[医学]{第五章\quad 疯狂的实验——医学}
\author[Yixf]{伊现富(Yi Xianfu)}
\institute[TIJMU]{天津医科大学(TIJMU)\\ 生物医学工程与技术学院}
\date{2017年3月}

\input{snippet/beamer_toc.tex}

\section{自体实验}
% 1-19,1802
\begin{frame}
  \frametitle{自体实验 | 黄热病}
  \begin{block}{疑问}
    黄热病能不能在人与人之间传播?(与黄热病人的接触有没有危险?)
  \end{block}
  \pause
  \begin{block}{实验}
    \begin{itemize}
      \item 1802年,斯塔宾斯\textbullet 弗斯(18岁),博士论文
      \item 狗/猫,黄热病人的呕吐物:喂食浸泡过的面包,敷在伤口上,注射到颈部静脉
      \item 他自己,呕吐物:敷在伤口上(20几个部位的重复),滴入眼睛,吸入蒸汽,吞下(制成的药片、稀释物)
      \item 他自己,其他体液或身体排泄物(血液、唾液、汗液、尿液)
    \end{itemize}
  \end{block}
  \pause
  \begin{block}{影响}
英雄之举影响甚微。实验主要揭示了黄热病无法通过一些方式传染,而人们想知道的却是:黄热病是怎么传播的?(蚊子!)
  \end{block}
\end{frame}

% 
\section{}

% 
\section{}

% 
\section{}

% 
\section{}


\section{史海撷华}
% 1-4,1600
\begin{frame}
  \frametitle{史海撷华 | 称盘上的生活}
  \begin{block}{实验}
    \begin{itemize}
      \item 圣多里奥(量化实验医学的鼻祖),《静态医学医疗术》
      \item 在圣多里奥的房子里,一切都悬于称上:床、工作台、椅子……
      \item 在称盘上度过的时间之长(30年)无人能及
      \item 记录自身体重的点滴变化:进食的食物重量,排泄的废物重量,……
    \end{itemize}
  \end{block}
  \pause
  \begin{block}{论断}
    \begin{itemize}
      \item 人们所排泄的大小便仅占所进食的食品重量的很小一部分(排汗,第一个测量这一重量的人)
    \end{itemize}
  \end{block}
\end{frame}

% 
\begin{frame}
  \frametitle{史海撷华 | }
\end{frame}

% 
\begin{frame}
  \frametitle{史海撷华 | }
\end{frame}

% 
\begin{frame}
  \frametitle{史海撷华 | }
\end{frame}

% 
\begin{frame}
  \frametitle{史海撷华 | }
\end{frame}

% 
\begin{frame}
  \frametitle{史海撷华 | }
\end{frame}

% 
\begin{frame}
  \frametitle{史海撷华 | }
\end{frame}

% 
\begin{frame}
  \frametitle{史海撷华 | }
\end{frame}

% 
\begin{frame}
  \frametitle{史海撷华 | }
\end{frame}


\input{snippet/class_tail}
