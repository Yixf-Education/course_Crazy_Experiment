\input{snippet/beamer_head}
\input{snippet/class_head}

%\includeonlyframes{current}

\title[医学]{第五章\quad 疯狂的实验——医学}
\author[Yixf]{伊现富(Yi Xianfu)}
\institute[TIJMU]{天津医科大学(TIJMU)\\ 生物医学工程与技术学院}
\date{2017年3月}

\input{snippet/beamer_toc.tex}

\section{自体实验}
% 1-19,1802
\begin{frame}
  \frametitle{自体实验 | 黄热病}
  \begin{block}{疑问}
    黄热病能不能在人与人之间传播?(与黄热病人的接触有没有危险?)
  \end{block}
  \pause
  \begin{block}{实验}
    \begin{itemize}
      \item 1802年,斯塔宾斯\textbullet 弗斯(18岁),博士论文
      \item 狗/猫,黄热病人的呕吐物:喂食浸泡过的面包,敷在伤口上,注射到颈部静脉
      \item 他自己,呕吐物:敷在伤口上(20几个部位的重复),滴入眼睛,吸入蒸汽,吞下(制成的药片、稀释物)
      \item 他自己,其他体液或身体排泄物(血液、唾液、汗液、尿液)
    \end{itemize}
  \end{block}
  \pause
  \begin{block}{影响}
英雄之举影响甚微。实验主要揭示了黄热病无法通过一些方式传染,而人们想知道的却是:黄热病是怎么传播的?(蚊子!)
  \end{block}
\end{frame}

% 1-197,1968
\begin{frame}
  \frametitle{自体实验 | 螨虫}
  \begin{block}{现象与疑问}
    \begin{itemize}
      \item 给猫治疗耳螨时,女房东和她女儿抱怨很痒。
      \item 耳螨能够传播给人类吗?
    \end{itemize}
  \end{block}
  \vspace{-0.5em}
  \pause
  \begin{block}{实验}
    \begin{itemize}
      \item 耳螨:取自猫耳,显微镜检查以确认是耳螨
      \item 实验:将大约1克混有耳螨的猫的耳垢放到被试(洛佩兹自己)的左耳里(先后三次实验)
    \end{itemize}
  \end{block}
  \vspace{-0.5em}
  \pause
  \begin{block}{结果}
    详细记录:声响、瘙痒、疼痛、耳垢……
  \end{block}
  \vspace{-0.5em}
  \pause
  \begin{block}{启示}
    \begin{itemize}
      \item 如果一项实验无法重复,实验的结果就算不上证据确凿。
      \item 1994年凭借其论文获得搞笑诺贝尔奖!
    \end{itemize}
  \end{block}
\end{frame}

% 1-197,1968
\begin{frame}
  \frametitle{自体实验 | 胃溃疡}
  \begin{block}{已知}
    \begin{itemize}
      \item 沃伦在针对胃黏膜炎患者的细胞实验中发现了一种不知名的细菌。
      \item 早在十九世纪就有研究者在人类和动物的胃中发现了螺旋杆菌。
    \end{itemize}
  \end{block}
  \pause
  \begin{block}{疑问}
    这些细菌与炎症有关系吗?
  \end{block}
  \pause
  \begin{block}{实验}
    马歇尔为一位病人施用抗菌素,结果那种不知名的病菌和胃炎同时消失了。
  \end{block}
  \pause
  \begin{block}{结论}
    细菌引发胃炎! vs. 细菌产生后,在胃的伤口处繁殖。
  \end{block}
\end{frame}

\begin{frame}
  \frametitle{自体实验 | 胃溃疡}
  \begin{block}{科赫法则}
    科赫在1882年就证明病菌的致病性提出了4项要求:
    \begin{enumerate}
      \item 细菌必须能够在每一个病例中找到
      \item 细菌必须能够在体外培养
      \item 培养的细菌能够使实验动物染病
      \item 细菌必须能再次从实验动物体内获得并培养
    \end{enumerate}
  \end{block}
  \pause
  \begin{block}{科赫法则一}
    沃伦和马歇尔在病人的胃壁上总能找到这种细菌。
  \end{block}
  \pause
  \begin{block}{科赫法则二}
    \begin{itemize}
      \item 一般来说,细菌在培养盘中繁殖的时间不会超过2天。
      \item 细菌培养48小时,一点儿增长的迹象都没有(大约30次无果的实验) $\Longrightarrow$ 培养皿在暖箱中放置5天,细菌数量奇迹般地增多了。
    \end{itemize}
  \end{block}
\end{frame}

\begin{frame}
  \frametitle{自体实验 | 胃溃疡}
  \begin{block}{科赫法则三、四}
    \begin{itemize}
      \item 给2只老鼠、2只小猪注射细菌,都没有染病的症状。
      \item 感染无法在动物身上得到证明,那就自己充当实验动物。
    \end{itemize}
  \end{block}
  \pause
  \begin{block}{自体实验}
    \begin{enumerate}
      \item 取样确认用于自身实验的胃是健康的
      \item 喝下收集自病人的细菌培养液
      \item 第十天,在胃出口取2份实验试样(一份检验,一份分离细菌进行培养)
    \end{enumerate}
  \end{block}
  \pause
  \begin{block}{结果}
    \begin{itemize}
      \item 马歇尔感染了胃炎。
      \item 培养液和10天前喝下去的是一样的。
    \end{itemize}
  \end{block}
\end{frame}

\begin{frame}
  \frametitle{自体实验 | 胃溃疡}
  \begin{block}{启示}
    \begin{itemize}
      \item 细菌(幽门螺旋杆菌)可以引发胃炎。
      \item 马歇尔在2005年获得诺贝尔奖。
      \item 他的实验掀起了用感染来解释其他疾病的新的热潮。
      \item 时至今日,借助细菌和病毒的手段,研究者对医治精神分裂症和心肌梗、风湿和糖尿病有了新的希望。迄今这些猜测只有少数得到了证实。
    \end{itemize}
  \end{block}
\end{frame}

% 1-121,1946
\section{感冒}
\begin{frame}
  \frametitle{感冒 | 实验准备}
  \begin{block}{观点与现象}
    \begin{itemize}
      \item 大众观点:寒冷本身引发感冒。
      \item 去往极地进行长期考察的研究人员并不感冒。
      \item 爱斯基摩人在最冷的冬天不会生病,却在有外来船只驶入港口时生病。
    \end{itemize}
  \end{block}
  \pause
  \begin{block}{实验准备:在实验之前排除一切由外部带入的感冒可能}
    \begin{itemize}
      \item 2人一组住进套房,接下来的10天里除室友以外,要与一切没有保护装置的人保持10米以上的距离
      \item 允许散步,但不得进入医院大楼和使用交通工具
      \item 医生和护士进行检查时身穿防护服、脸配面具
      \item 每日3餐被装入保温容器,按时送到套房门前
    \end{itemize}
  \end{block}
\end{frame}

\begin{frame}
  \frametitle{感冒 | 实验}
  \begin{block}{实验}
    \begin{itemize}
      \item 将被试分成3组,每组6人
      \item 第一组接受鼻腔注入(将一位感冒患者的鼻腔分泌物过滤并稀释后滴入鼻中)
      \item 第二组接受冷却处理(先热水浸泡,后在通风的走廊里坚持带上半个小时,其余时间必须穿着湿袜子)
      \item 第三组同时施用冷却处理和鼻腔注入两种办法
    \end{itemize}
  \end{block}
  \pause
  \begin{block}{结果}
    \begin{itemize}
      \item 既受病毒感染又遭降温处理的人中有4人生病
      \item 只受病毒感染的人中有2人生病
      \item 单纯受凉并未造成任何人感冒
    \end{itemize}
  \end{block}
\end{frame}

\begin{frame}
  \frametitle{感冒 | 重复实验}
  \begin{block}{重复}
      20世纪50、60年代,上百名被试,一系列实验
  \end{block}
  \pause
  \begin{block}{结果}
    \begin{itemize}
      \item 一致:没有显示感冒与之前的受寒有何关联
      \item 不一致:只受病毒感染的小组中的感冒人数比既受感染又遭寒冷的小组中的感冒人数翻了一番
    \end{itemize}
  \end{block}
  \pause
  \begin{block}{启示}
    \begin{itemize}
      \item 科学在面对大众观念时处境艰难!
      \item 为什么感冒的冬季发病率高于夏季?(至今尚不十分明确)
      \item 冬季感冒大量发病与虚弱的免疫系统及房间的干燥空气并无关系。
      \item 更可能的情况是:冬季房间通风不良,冬季太阳照射相对较弱(紫外线可以杀灭病原)。
    \end{itemize}
  \end{block}
\end{frame}

% 
\section{}

% 
\section{}

% 
\section{}


\section{史海撷华}
% 1-4,1600
\begin{frame}
  \frametitle{史海撷华 | 称盘上的生活}
  \begin{block}{实验}
    \begin{itemize}
      \item 圣多里奥(量化实验医学的鼻祖),《静态医学医疗术》
      \item 在圣多里奥的房子里,一切都悬于称上:床、工作台、椅子……
      \item 在称盘上度过的时间之长(30年)无人能及
      \item 记录自身体重的点滴变化:进食的食物重量,排泄的废物重量,……
    \end{itemize}
  \end{block}
  \pause
  \begin{block}{论断}
    \begin{itemize}
      \item 人们所排泄的大小便仅占所进食的食品重量的很小一部分(排汗,第一个测量这一重量的人)
    \end{itemize}
  \end{block}
\end{frame}

% 
\begin{frame}
  \frametitle{史海撷华 | }
\end{frame}

% 
\begin{frame}
  \frametitle{史海撷华 | }
\end{frame}

% 
\begin{frame}
  \frametitle{史海撷华 | }
\end{frame}

% 
\begin{frame}
  \frametitle{史海撷华 | }
\end{frame}

% 
\begin{frame}
  \frametitle{史海撷华 | }
\end{frame}

% 
\begin{frame}
  \frametitle{史海撷华 | }
\end{frame}

% 
\begin{frame}
  \frametitle{史海撷华 | }
\end{frame}

% 
\begin{frame}
  \frametitle{史海撷华 | 其他}
  \begin{block}{真正的疯狂}
    \begin{itemize}
      % 1-37,1889
      \item 返老还童——注射过滤后的浆液(捣碎的睾丸+蒸馏水),安慰剂效应、荷尔蒙疗法
      % 1-47,1899
      \item 脊椎麻醉——医生助手齐上阵,频繁详细的实验记录
      \item 注射曼巴蛇毒——眼镜蛇 vs. 曼巴蛇,错误的估计、巨大的代价
      \item 
      \item 
      \item 
      \item 
      \item 
      \item 
    \end{itemize}
  \end{block}
\end{frame}


\input{snippet/class_tail}
