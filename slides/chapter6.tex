\input{snippet/beamer_head}
\input{snippet/class_head}

%\includeonlyframes{current}

\title[其他]{第五章\quad 疯狂的实验——其他}
\author[Yixf]{伊现富(Yi Xianfu)}
\institute[TIJMU]{天津医科大学(TIJMU)\\ 生物医学工程与技术学院}
\date{2017年4月}

\input{snippet/beamer_toc.tex}

% 1-221,1970
\section{福克斯博士效应}
\begin{frame}
  \frametitle{福克斯博士效应 | 实验}
  \begin{block}{问题}
 是否可以用炫人耳目的演讲技巧欺骗一组专家,使他们注意不到内容的荒唐无稽? 
  \end{block}
  \pause
  \begin{block}{实验}
    \begin{itemize}
      \item 福克斯:“将数学科学应用于人类行为的权威”(演员)
      \item 报告:《培训医生过程中所应用的数学博弈理论》
      \item 听众:经验丰富的教育学家
      \item 策划:报告充满了含混不清的言语、凭空捏造的词句和自相矛盾的结论,借助大量的幽默和参阅提示把这些胡言乱语的内容组织起来;在提问环节,使用高超巧妙的方式予以回答
    \end{itemize}
  \end{block}
\end{frame}

\begin{frame}
  \frametitle{福克斯博士效应 | 结果}
  \begin{block}{结果}
    \begin{itemize}
      \item 一小时后报告正文结束,大家积极提问。
      \item 全部10位听众都表示这次报告启迪了他们的思考。
      \item 9位认为福克斯材料整理清楚有趣、内容介绍妙趣横生并且穿插了丰富的阐释性事例。
      \item 播放报告录像,有一位居然认为他以前读过福克斯的论文。
      \item 得知福克斯的真实身份后,有几位听众想要询问更多的文献信息。
    \end{itemize}
  \end{block}
  \pause
  \begin{block}{福克斯博士效应}
    报告的风格完全可以演示贫乏的内容。
  \end{block}
\end{frame}

\begin{frame}
  \frametitle{福克斯博士效应 | 启示}
  \begin{block}{启示}
    \begin{itemize}
      \item 课程评估的说服力值得怀疑:学生们在问卷中写下的课程评价可能无非是他们自己的满意感和“学到了东西的错觉”。
      \item 尽管报告是一个一无所云的骗局,但它的风格唤起了听众对题目的兴趣。 $\Longrightarrow$ 建议人们创新方法提高学生的学习动力。
      \item 如果一个演员可以成为高明的老师,为什么不能成为出色的议员甚至优秀的总统呢? $\Longrightarrow$ 1980年,里根成为美国总统!(他的演说风格高明而极具说服力,被媒体誉为“伟大的沟通者”,美国人心目中最伟大的总统之一。)
    \end{itemize}
  \end{block}
\end{frame}

% 
\section{}

% 
\section{}

% 
\section{}


\section{史海撷华}
% 2-23,1874
\begin{frame}
  \frametitle{史海撷华 | 创伤弹道学}
  \begin{block}{疑问}
    枪击究竟为何能够致死?
  \end{block}
  \pause
  \begin{block}{猜测}
    \begin{enumerate}
      \item “碎片”说:极高的温度使子弹熔化,导致许多碎片从子弹上剥落下来。
      \item “离心力”说:旋转的子弹带有离心力,折断了皮肉。
      \item “压力”说:子弹侵入肌肉以及柔软部位后产生了一种压力。
    \end{enumerate}
  \end{block}
  \pause
  \begin{block}{观察}
    \begin{itemize}
      \item 子弹飞出体外的地方、破掉的皮肉并没有扭转出涡旋式的形状——“离心力”的说法并不准确。
      \item 如果子弹没有击中骨骼就不会留下什么碎片——“碎片”的说法也不可靠。
    \end{itemize}
  \end{block}
\end{frame}

\begin{frame}
  \frametitle{史海撷华 | 创伤弹道学}
  \begin{block}{实验}
    \begin{itemize}
      \item 主持:科赫尔(首位获得诺贝尔医学奖的外科医生,1909年)
      \item 武器:口径为10.4、11毫米的“维特尔利”、“夏斯波”
      \item 靶子:杉木板、书、装满沙子的猪膀胱、塞满土豆泥的人类颅骨、用布裹住的尸体、石板、铁皮罐、玻璃板、肝脏……
    \end{itemize}
  \end{block}
  \vspace{-0.3em}
  \pause
  \begin{block}{结果}
     子弹打在空的颅骨上——两个洞;打在用土豆泥填充的颅骨上——开花。
  \end{block}
  \vspace{-0.3em}
  \pause
  \begin{block}{启示}
    \begin{itemize}
      \item 枪击造成的流体静力冲击才是摧毁人体组织的元凶。
      \item 为“鲁宾弹”的设计提供理论基础;开创“创伤弹道学”研究先河。
      \item 战争的目的不是尽量消灭更多人的生命,而是“把一个拥有较强战斗能力的敌对者变成一个需要照料的病人”。
    \end{itemize}
  \end{block}
\end{frame}

% 1-193,1966
\begin{frame}
  \frametitle{史海撷华 | 搭车技巧 | 弱不禁风}
  \begin{block}{初衷}
    研究汽车司机对待拦车搭乘者的态度。
  \end{block}
  \pause
  \begin{block}{实验}
    男生,用4天时间站在一条4车道的马路边尝试搭车,有时膝盖裹着绷带并拄着拐杖,有时则没有。
  \end{block}
  \pause
  \begin{block}{结果}
    绑绷带拄拐杖时获得了至少双重的搭车机会 $\Longrightarrow$ 搭车技巧之一:要弱不禁风!
  \end{block}
\end{frame}

% 1-231,1971
\begin{frame}
  \frametitle{史海撷华 | 搭车技巧 | 是个女的}
  \begin{block}{《搭乘的成功率》}
    \begin{itemize}
      \item 研究男女在穿着不同服装时,站在路边等待搭乘的成功率。
      \item 研究不同人数和搭配对司机态度的影响。
    \end{itemize}
  \end{block}
  \pause
  \begin{block}{结论}
    \begin{itemize}
      \item 衣着邋遢的比衣着整洁干净的难以搭车。
      \item 男人比女人更难搭车。
      \item 2个男人是最难搭车的。
      \item 单独一个男人搭车的成功率根一对情侣相差无几。
      \item 成功率最高的是2个女人在一起时。
    \end{itemize}
  \end{block}
\end{frame}

% 1-231,1971
\begin{frame}
  \frametitle{史海撷华 | 搭车技巧 | 更多技巧}
  \begin{block}{目光交流}
    \begin{itemize}
      \item 当被注视时,加州帕罗奥多的600辆车中,有40辆车停了下来。
      \item 如果司机和搭乘者之间没有目光交流,仅有18辆车会停下来。
    \end{itemize}
  \end{block}
  \pause
  \begin{block}{丰胸}
在西雅图进行的一次实验中,带着垫厚的胸罩(加上5厘米的外延)的女士获得了与不这样做时相比双倍的搭乘机会。 
  \end{block}
\end{frame}

% 1-254,1976 
\begin{frame}
  \frametitle{史海撷华 | 剃须刀教学法}
  \begin{block}{实验室结论}
    \begin{itemize}
      \item “对作为客体的人的稳定的刺激特征”(胡子)会在对人性格的判断上产生深入的影响。
      \item 脸上的一个细小变化,就会让观察者获得不同的印象。
    \end{itemize}
  \end{block}
  \vspace{-0.5em}
  \pause
  \begin{block}{疑问}
      实验室研究中得出的上述结论在现实中是否成立呢?
      %《胡子的魔术——关于长胡子的高校教员对学生所产生的影响的研究》
  \end{block}
  \vspace{-0.5em}
  \pause
  \begin{block}{实验}
    \begin{itemize}
      \item 不留胡子(2学期)——留胡子(1学期)——不留胡子(1学期)
      \item 学期开始见面10分钟后,填写评价问卷
    \end{itemize}
  \end{block}
  \vspace{-0.5em}
  \pause
  \begin{block}{结果}
留胡子:不够有目标、不够专心、不够友善、不够坚强、不够理智、不够聪明……;看起来不拘束、比较随意……
  \end{block}
\end{frame}

% 2-68,1932 
\begin{frame}
  \frametitle{史海撷华 | 挠痒痒 | 与笑的关联}
  \begin{block}{现象}
小孩子学会在挠痒痒的过程中发笑,是因为挠痒痒一般都发生于玩闹的情境中,而小孩子在玩耍时,本来就是会笑的,笑未必和挠痒痒直接相关。
  \end{block}
  \pause
  \begin{block}{疑问}
    \begin{itemize}
      \item 我们被挠痒痒时为什么会笑?
      \item 挠痒痒和笑之间的关联是天生的吗?
    \end{itemize}
  \end{block}
  \pause
  \begin{block}{实验}
    \begin{itemize}
      \item 克拉伦斯\textbullet 柳巴,自己的第4和第5个孩子
      \item 实验:用纸板面具遮住脸,给孩子挠痒痒
      \item 对照:其余阶段一次都不许给孩子挠痒痒
    \end{itemize}
  \end{block}
  \pause
  \begin{block}{结果与结论}
    \begin{itemize}
      \item 31周大时,第一次在被挠痒痒时自发地笑了起来。
      \item 挠痒痒和笑似乎有着天生的关联。
    \end{itemize}
  \end{block}
\end{frame}

% 2-168,1970
\begin{frame}
  \frametitle{史海撷华 | 挠痒痒 | 足部挠痒机}
  \begin{block}{疑问}
    人类为何不能对自己挠痒并让自己发笑?
  \end{block}
  \vspace{-0.5em}
  \pause
  \begin{block}{实验}
      足部挠痒机(将挠痒刺激标准化),30名学生,控制方式:自行 vs. 陌生人 vs. 他人控制+被试手握
  \end{block}
  \vspace{-0.5em}
  \pause
  \begin{block}{结果}
    \begin{itemize}
      \item 与自行控制相比,一个陌生人控制操纵杆会让他们明显感到刺痒。
      \item 自己手握操纵杆但由他人控制时,发痒敏感度虽然有所减弱,却仍然比自己控制时要强。
    \end{itemize}
  \end{block}
  \vspace{-0.5em}
  \pause
  \begin{block}{结论——《自然》}
    \begin{itemize}
      \item 得知被挠痒的时间和位置,并不足以完全压制发痒感,除非本人掌握控制权。想要压制发痒感,被挠痒者本人必须掌握控制权。
      \item “人类会什么会怕痒”这一根本问题仍然是个谜。
    \end{itemize}
  \end{block}
\end{frame}

% 2-231,1994
\begin{frame}
  \frametitle{史海撷华 | 挠痒痒 | 机器挠痒痒}
  \begin{block}{疑问}
    机器能挠痒痒吗?如果挠痒痒方式完全相同,只是一个来自人类,一个来自机器,是否会造成什么区别?
  \end{block}
  \pause
  \begin{block}{实验}
    \begin{itemize}
      \item 一台挠痒痒机器人(完全不起作用),被试(塞住耳朵、蒙上眼睛)
      \item 被试被挠2次痒:1次由人来挠,1次由机器来挠(事实上2次都是由一人完成的)
      \item 比较被试者在被人以及被机器挠痒痒时发笑的强度(面部视频录像——被试自我评价)
    \end{itemize}
  \end{block}
  \pause
  \begin{block}{结果与结论}
    \begin{itemize}
      \item 发笑的强度始终相同,与被人还是被“机器”挠痒痒并无关联。
      \item 挠痒痒时的发笑并没有任何社会性的特征,只是一种反射行为。
    \end{itemize}
  \end{block}
\end{frame}

% 2-22,1822
\begin{frame}
  \frametitle{史海撷华 | 圣经 | 鬣狗}
  \begin{block}{缘起}
采石场的工人们发现了一个洞穴,里面堆满了老虎、牡鹿、熊、马、大象、犀牛、河马和鬣狗的骨头。
  \end{block}
  \vspace{-0.5em}
  \pause
  \begin{block}{解释与疑问}
    \begin{itemize}
      \item 解释:教士借助《圣经》故事——野兽们被“大洪水”冲进洞穴。
      \item 疑问:洞里没有沙石的踪迹。大型动物是怎样通过狭窄的洞口挤入洞中的呢?
    \end{itemize}
  \end{block}
  \vspace{-0.5em}
  \pause
  \begin{block}{猜测与实验}
    \begin{itemize}
      \item 观察:骨头上有被啃噬的痕迹,而且牙印与鬣狗牙齿完全吻合。
      \item 推断:鬣狗将觅得的腐尸拖进洞里慢慢享用。
      \item 实验:从非洲南部引进鬣狗,让它啃食骨头。
      \item 结果:不吃的部位都能在洞穴里找到,吃掉的部位证实洞穴里没有的。实验用的骨头和洞穴里的骨头的断裂状况完全一样。
    \end{itemize}
  \end{block}
\end{frame}

% 2-70,1932
% 2-186,1984
\begin{frame}
  \frametitle{史海撷华 | 圣经 | 十字架刑}
  \begin{block}{疑问}
    \begin{itemize}
      \item 人的身体不可能只靠三只钉子就挂住?
      \item 钉在十字架上死去的生理过程。
    \end{itemize}
  \end{block}
  \pause
  \begin{block}{实验}
    \begin{itemize}
      \item 1个十字架、1把锤头、3只钉子、1具尸体、12只“刚切下来的手臂”、若干只脚
      \item 一座2.3米高的(带有腕带的)十字架
    \end{itemize}
  \end{block}
  \pause
  \begin{block}{结果}
    \begin{itemize}
      \item 被钉上十字架的耶稣最终死于窒息?
      \item 耶稣死于心脏停跳和呼吸中断,是由大量失血和创伤性休克所致。
    \end{itemize}
  \end{block}
\end{frame}

% 1-292,1998
\begin{frame}
  \frametitle{史海撷华 | 圣经 | 耶利哥的扬声器}
  \begin{block}{耶利哥的扬声器}
    \begin{itemize}
      \item 7个祭祀在约柜前吹响他们的羊角,使得耶利哥的城墙倾陷——《圣经\textbullet 约书亚记》
      \item 美国教育节目《学习频道》,加利福尼亚的怀勒实验室,怀勒WAS 3000扬声器(相当于10000个扬声喇叭)
      \item 结果:经过6分钟的噪声作用,砂浆真的开始破碎,小墙分崩离析
      \item 结论:(众所周知的事实)声音真的可以产生破坏
      \item 事实:耶利哥所属的迦南城根本没有设防……
    \end{itemize}
  \end{block}
\end{frame}

% 2-114,1960
\begin{frame}
  \frametitle{史海撷华 | 眼睛}
  \begin{block}{“心灵之窗”}
    \begin{itemize}
      \item 大脑从事某些特定活动时,瞳孔大小会发生怎样的变化?
      \item 被观看者的瞳孔大小会对观看者产生什么影响?
    \end{itemize}
  \end{block}
  \pause
  \begin{block}{灵感}
    某天晚上,妻子观察到他在翻看一本动物画册时,瞳孔突然放大。
  \end{block}
  \pause
  \begin{block}{实验}
    \begin{itemize}
      \item 助理,“招贴画女郎”实验(在一大叠风景图片中混入一张“香艳热辣”的美女照片)——瞳孔直径测量
      \item 2男2女,不同图片(婴儿、裸体男/女性、残疾儿童、现代艺术……)
      \item 男士,一位女性的2张照片(唯一区别在于瞳孔大小)
    \end{itemize}
  \end{block}
\end{frame}

\begin{frame}
  \frametitle{史海撷华 | 眼睛}
  \begin{block}{结果}
    \begin{itemize}
      \item 图片内容为婴儿、母亲怀抱婴儿、裸体男性时,女性被试者的瞳孔明显放大;男性被试者对裸体女性的反应尤为强烈。
      \item 男性被试者观看瞳孔放大的那张照片时,自己的瞳孔也放大得特别明显。
      \item 只要观看者产生了关注的兴趣,瞳孔就会放大;大脑在努力加工信息时,瞳孔也会方法。
      \item 检测人类思想活动的终极手段(性取向,预测不同产品的市场价值)?
    \end{itemize}
  \end{block}
  \pause
  \begin{block}{未解之谜}
    \begin{itemize}
      \item 瞳孔为什么会随着大脑活动而运动呢?
      \item 这种现象蕴含着什么深刻的意义吗?
      \item 它只是大脑在处理其他事情的时候产生的副产品吗?
    \end{itemize}
  \end{block}
\end{frame}

% 2-125,1962
\begin{frame}
  \frametitle{史海撷华 | 极度恐惧}
  \begin{block}{疑问}
    极度恐惧如何影响脑力活动效率?
  \end{block}
  \vspace{-0.3em}
  \pause
  \begin{block}{实验}
    \begin{itemize}
      \item 20个新兵,阅读紧急情况指示,飞机迫降,填写调查问卷
      \item 新兵所在位置误划为炮火攻击目标,炮击/放射性污染/森林大火,维修无线电设备
    \end{itemize}
  \end{block}
  \vspace{-0.3em}
  \pause
  \begin{block}{结果}
    \begin{itemize}
      \item 20人中只有5人没有在“紧急情况”下惊慌失措。其他人都“对死亡或受伤产生了不同程度的恐惧”。
      \item 填写问卷的过程张,惶恐情绪体现得最为明显,尤其是在回忆安全指令时,很多人忘记了近乎一般的内容。
      \item 只有炮击场景会分散新兵修理无线电设备的注意力,另外2种场景(放射性污染和森林大货)并没有影响他们的效率。
    \end{itemize}
  \end{block}
\end{frame}

% 2-135,1964
\begin{frame}
  \frametitle{史海撷华 | 厌恶疗法}
  \begin{block}{问题}
    如何消除酒瘾?如何对付过度贪玩、暴食症或性取向异常?
  \end{block}
  \pause
  \begin{block}{方法}
在酒徒的杯子里放几只蜘蛛(不愉快的刺激),注射药物导致呼吸停止(产生极大的恐惧),实施电击、让病人闻刺鼻的气味或者服用引起呕吐的药物。
  \end{block}
  \pause
  \begin{block}{厌恶疗法}
    将欲戒除的行为和不愉快的刺激进行结合。
  \end{block}
  \pause
  \begin{block}{启示}
    以实验为目的,让人毫不知情地陷入极度恐惧,是一件不可容忍的事情。
  \end{block}
\end{frame}

% 2-141,1965
\begin{frame}
  \frametitle{史海撷华 | 加芬克尔化}
  \begin{block}{缘起}
    \begin{itemize}
      \item 人们在说话时,信息表达往往很不完整。令人吃惊的是,它没有给任何人带来困扰。精确细致的表达或者持续不断的追问反而会让人不胜其烦。
      \item 无障碍的交流大多是建立在语言含混的基础上的——这听起来像是一个悖论。
    \end{itemize}
  \end{block}
  \pause
  \begin{block}{启示}
    \begin{itemize}
      \item 人们从含义不明的句子中整合出固定的意义,加芬克尔将这种策略称为“俗民方法学”。
      \item 我们的交流非常依赖共同的背景知识和隐含的猜测。——“破坏性实验”
      \item 有些美国人会将有意打破默认的文化规则的行为称为“加芬克尔化”。
    \end{itemize}
  \end{block}
\end{frame}

% 
\begin{frame}
  \frametitle{史海撷华 | }
\end{frame}

% 
\begin{frame}
  \frametitle{史海撷华 | }
\end{frame}

% 
\begin{frame}
  \frametitle{史海撷华 | }
\end{frame}


\input{snippet/class_tail}
