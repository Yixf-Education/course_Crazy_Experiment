\input{snippet/beamer_head}
\input{snippet/class_head}

%\includeonlyframes{current}

\title[其他]{第五章\quad 疯狂的实验——其他}
\author[Yixf]{伊现富(Yi Xianfu)}
\institute[TIJMU]{天津医科大学(TIJMU)\\ 生物医学工程与技术学院}
\date{2017年4月}

\input{snippet/beamer_toc.tex}

% 1-221,1970
\section{福克斯博士效应}
\begin{frame}
  \frametitle{福克斯博士效应 | 实验}
  \begin{block}{问题}
 是否可以用炫人耳目的演讲技巧欺骗一组专家,使他们注意不到内容的荒唐无稽? 
  \end{block}
  \pause
  \begin{block}{实验}
    \begin{itemize}
      \item 福克斯:“将数学科学应用于人类行为的权威”(演员)
      \item 报告:《培训医生过程中所应用的数学博弈理论》
      \item 听众:经验丰富的教育学家
      \item 策划:报告充满了含混不清的言语、凭空捏造的词句和自相矛盾的结论,借助大量的幽默和参阅提示把这些胡言乱语的内容组织起来;在提问环节,使用高超巧妙的方式予以回答
    \end{itemize}
  \end{block}
\end{frame}

\begin{frame}
  \frametitle{福克斯博士效应 | 结果}
  \begin{block}{结果}
    \begin{itemize}
      \item 一小时后报告正文结束,大家积极提问。
      \item 全部10位听众都表示这次报告启迪了他们的思考。
      \item 9位认为福克斯材料整理清楚有趣、内容介绍妙趣横生并且穿插了丰富的阐释性事例。
      \item 播放报告录像,有一位居然认为他以前读过福克斯的论文。
      \item 得知福克斯的真实身份后,有几位听众想要询问更多的文献信息。
    \end{itemize}
  \end{block}
  \pause
  \begin{block}{福克斯博士效应}
    报告的风格完全可以演示贫乏的内容。
  \end{block}
\end{frame}

\begin{frame}
  \frametitle{福克斯博士效应 | 启示}
  \begin{block}{启示}
    \begin{itemize}
      \item 课程评估的说服力值得怀疑:学生们在问卷中写下的课程评价可能无非是他们自己的满意感和“学到了东西的错觉”。
      \item 尽管报告是一个一无所云的骗局,但它的风格唤起了听众对题目的兴趣。 $\Longrightarrow$ 建议人们创新方法提高学生的学习动力。
      \item 如果一个演员可以成为高明的老师,为什么不能成为出色的议员甚至优秀的总统呢? $\Longrightarrow$ 1980年,里根成为美国总统!(他的演说风格高明而极具说服力,被媒体誉为“伟大的沟通者”,美国人心目中最伟大的总统之一。)
    \end{itemize}
  \end{block}
\end{frame}

% 
\section{}

% 
\section{}

% 
\section{}

% 
\section{}


\section{史海撷华}
% 1-193,1966
\begin{frame}
  \frametitle{史海撷华 | 搭车技巧 | 弱不禁风}
  \begin{block}{初衷}
    研究汽车司机对待拦车搭乘者的态度。
  \end{block}
  \pause
  \begin{block}{实验}
    男生,用4天时间站在一条4车道的马路边尝试搭车,有时膝盖裹着绷带并拄着拐杖,有时则没有。
  \end{block}
  \pause
  \begin{block}{结果}
    绑绷带拄拐杖时获得了至少双重的搭车机会 $\Longrightarrow$ 搭车技巧之一:要弱不禁风!
  \end{block}
\end{frame}

% 1-231,1971
\begin{frame}
  \frametitle{史海撷华 | 搭车技巧 | 是个女的}
  \begin{block}{《搭乘的成功率》}
    \begin{itemize}
      \item 研究男女在穿着不同服装时,站在路边等待搭乘的成功率。
      \item 研究不同人数和搭配对司机态度的影响。
    \end{itemize}
  \end{block}
  \pause
  \begin{block}{结论}
    \begin{itemize}
      \item 衣着邋遢的比衣着整洁干净的难以搭车。
      \item 男人比女人更难搭车。
      \item 2个男人是最难搭车的。
      \item 单独一个男人搭车的成功率根一对情侣相差无几。
      \item 成功率最高的是2个女人在一起时。
    \end{itemize}
  \end{block}
\end{frame}

% 1-231,1971
\begin{frame}
  \frametitle{史海撷华 | 搭车技巧 | 更多技巧}
  \begin{block}{目光交流}
    \begin{itemize}
      \item 当被注视时,加州帕罗奥多的600辆车中,有40辆车停了下来。
      \item 如果司机和搭乘者之间没有目光交流,仅有18辆车会停下来。
    \end{itemize}
  \end{block}
  \pause
  \begin{block}{丰胸}
在西雅图进行的一次实验中,带着垫厚的胸罩(加上5厘米的外延)的女士获得了与不这样做时相比双倍的搭乘机会。 
  \end{block}
\end{frame}

% 
\begin{frame}
  \frametitle{史海撷华 | }
\end{frame}

% 
\begin{frame}
  \frametitle{史海撷华 | }
\end{frame}

% 
\begin{frame}
  \frametitle{史海撷华 | }
\end{frame}

% 
\begin{frame}
  \frametitle{史海撷华 | }
\end{frame}

% 
\begin{frame}
  \frametitle{史海撷华 | }
\end{frame}

% 
\begin{frame}
  \frametitle{史海撷华 | }
\end{frame}

% 
\begin{frame}
  \frametitle{史海撷华 | }
\end{frame}

% 
\begin{frame}
  \frametitle{史海撷华 | }
\end{frame}

% 
\begin{frame}
  \frametitle{史海撷华 | }
\end{frame}


\input{snippet/class_tail}
