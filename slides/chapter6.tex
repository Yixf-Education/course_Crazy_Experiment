\input{snippet/beamer_head}
\input{snippet/class_head}

%\includeonlyframes{current}

\title[其他]{第五章\quad 疯狂的实验——其他}
\author[Yixf]{伊现富(Yi Xianfu)}
\institute[TIJMU]{天津医科大学(TIJMU)\\ 生物医学工程与技术学院}
\date{2018年4月}

\input{snippet/beamer_toc.tex}

% 1-221,1970
\section{福克斯博士效应}
\begin{frame}
  \frametitle{福克斯博士效应 | 实验}
  \begin{block}{问题}
 是否可以用炫人耳目的演讲技巧欺骗一组专家,使他们注意不到内容的荒唐无稽? 
  \end{block}
  \pause
  \begin{block}{实验}
    \begin{itemize}
      \item 福克斯:“将数学科学应用于人类行为的权威”(演员)
      \item 报告:《培训医生过程中所应用的数学博弈理论》
      \item 听众:经验丰富的教育学家
      \item 策划:报告充满了含混不清的言语、凭空捏造的词句和自相矛盾的结论,借助大量的幽默和参阅提示把这些胡言乱语的内容组织起来;在提问环节,使用高超巧妙的方式予以回答
    \end{itemize}
  \end{block}
\end{frame}

\begin{frame}
  \frametitle{福克斯博士效应 | 结果}
  \begin{block}{结果}
    \begin{itemize}
      \item 一小时后报告正文结束,大家积极提问。
      \item 全部10位听众都表示这次报告启迪了他们的思考。
      \item 9位认为福克斯材料整理清楚有序、内容介绍妙趣横生并且穿插了丰富的阐释性事例。
      \item 播放报告录像,有一位居然认为他以前读过福克斯的论文。
      \item 得知福克斯的真实身份后,有几位听众想要询问更多的文献信息。
    \end{itemize}
  \end{block}
  \pause
  \begin{block}{\alert{福克斯博士效应}}
    报告的风格完全可以掩饰贫乏的内容。
  \end{block}
\end{frame}

\begin{frame}
  \frametitle{福克斯博士效应 | 启示}
  \begin{block}{启示}
    \begin{itemize}
      \item \alert{课程评估的说服力值得怀疑}:学生们在问卷中写下的课程评价可能无非是他们自己的满意感和“学到了东西的错觉”。(授课远远不止是让学生们高兴。)
      \item 尽管报告是一个一无所云的骗局,但它的风格唤起了听众对题目的兴趣。 $\Longrightarrow$ 建议人们\alert{创新方法提高学生的学习动力}。
      \item 如果一个演员可以成为高明的老师,为什么不能成为出色的议员甚至优秀的总统呢?
    \end{itemize}
  \end{block}
\end{frame}

\begin{frame}
  \frametitle{福克斯博士效应 | 启示}
  \begin{figure}
    \centering
    \includegraphics[width=0.5\textwidth]{c6_Reagan_01.jpg}\quad
    \includegraphics[width=0.33\textwidth]{c6_Reagan_02.jpg}
  \end{figure}
  \pause
  \begin{block}{里根}
 1980年,里根成为美国总统!(他的演说风格高明而极具说服力,被媒体誉为“伟大的沟通者”,美国人心目中最伟大的总统之一。) 
  \end{block}
\end{frame}

% 2-196,1988
\section{暴力的黑色}
\begin{frame}
  \frametitle{暴力的黑色 | 缘起}
  \begin{block}{“经验”}
    \begin{itemize}
      \item 身穿黑色球衣的球队比身穿其他颜色球衣的球队打球更具攻击性,犯规次数更多。
      \item 遛狗经验:人们避开黑狗走路;人们绕道而行时,黑狗也变得放肆起来。
    \end{itemize}
  \end{block}
  \pause
  \begin{block}{猜测}
    黑色不仅引发了人们的恐惧,同时也激起了人/狗的攻击性。
  \end{block}
  \pause
  \begin{block}{疑问}
    \begin{itemize}
      \item 生活中的观察/感觉是否符合实际情况?
      \item 观察者觉得黑色更具攻击性?
      \item 黑色让自己变得更具攻击性?
    \end{itemize}
  \end{block}
\end{frame}

\begin{frame}
  \frametitle{暴力的黑色 | 实验一}
  \begin{block}{疑问}
      生活中的观察/感觉是否符合实际情况?
  \end{block}
  \pause
  \begin{block}{实验一}
      向25名参与者展示冰球和美式橄榄球球队的球衣;更换球衣
  \end{block}
  \vspace{-0.5em}
  \pause
  \begin{block}{结果}
    \begin{itemize}
      \item 黑色球衣更具攻击性;黑衣球队在受罚方面都名列前茅。
      \item 球队更换黑色球衣以来,受罚时间明显增长。
    \end{itemize}
  \end{block}
\end{frame}

\begin{frame}
  \frametitle{暴力的黑色 | 实验二}
  \begin{block}{疑问}
    是他们身穿黑色的时候更具攻击性,还是裁判受到错觉的干扰,以为他们更具攻击性呢?
  \end{block}
  \pause
  \begin{block}{实验二}
    \begin{itemize}
      \item 2张图片,展现一模一样的比赛场景,只是“哪支球队身穿黑色球衣”发生了变化。
      \item 对抗较为频繁的比赛录像,用投影仪描画出球员轮廓,复印后将进攻方的球衣一次涂成黑色,一次涂成红色。
    \end{itemize}
  \end{block}
  \pause
  \begin{block}{结果}
    \begin{itemize}
      \item 被试者面对这些图片,根本无从评判。
      \item 要评判进攻行为,需要生动的画面。
    \end{itemize}
  \end{block}
\end{frame}

\begin{frame}
  \frametitle{暴力的黑色 | 实验三}
  \begin{block}{实验三}
    \begin{itemize}
      \item 请朋友们2箱啤酒,前提是——穿上美式橄榄球球衣、模拟几次激烈的比赛场景。
      \item 从照片中选出几个相同的比赛场景,展示给美式橄榄球球迷和裁判。
    \end{itemize}
  \end{block}
  \pause
  \begin{block}{结果}
    \begin{itemize}
      \item 裁判会对身穿黑衣的球队给出更重的惩罚。
      \item 球迷们也感觉这支球队打得更具攻击性。
    \end{itemize}
  \end{block}
  \pause
  \begin{block}{猜测}
    2种效应同时出现?(身穿黑衣的人不仅被人认为更具攻击性,而且真的变得更具攻击性。)
  \end{block}
\end{frame}

\begin{frame}
  \frametitle{暴力的黑色 | 实验四}
  \begin{block}{实验四}
    编个理由,请72名被试者穿上黑色或者白色的T恤衫,要求他们从一张列有12种比赛的表格中选出想要参与的5场。
  \end{block}
  \pause
  \begin{block}{结果}
  身穿黑衣的人选择了攻击性更强的比赛。
  \end{block}
  \pause
  \begin{block}{启示}
    人们会受到看似毫无意义的表象的影响。(与直觉相反!)
  \end{block}
  \pause
  \begin{block}{扩展}
    黑色更具攻击性,有没有能提升胜率的颜色?
  \end{block}
\end{frame}

\begin{frame}
  \frametitle{暴力的黑色 | 红色}
  \begin{block}{奥运会}
    \begin{itemize}
      \item 2004年,夏季奥运会,4个对抗性运动项目的比赛,双方抽签获得蓝色和红色的服装。
      \item 身穿红色的运动员赢面更大。
    \end{itemize}
  \end{block}
  \pause
  \begin{block}{欧洲杯}
    \begin{itemize}
      \item 2004年,欧洲杯,身穿红色球衣的球队更常以胜者的身份走出球场。
      \item 克罗地亚、捷克共和国、英格兰、拉脱维亚和西班牙在身穿红色球衣时进球更多,比身穿其他颜色球衣时平均多进0.97个球。
    \end{itemize}
  \end{block}
  \pause
  \begin{block}{启示}
    \begin{itemize}
      \item 红色为什么会有这样的效果,目前尚不清楚。
      \item 在许多动物眼里,红色都是强势的标志。
    \end{itemize}
  \end{block}
\end{frame}

% 2-255,1999
\section{主场优势}
\begin{frame}
  \frametitle{主场优势 | 简介}
  \begin{block}{主场优势——证明很简单,解释却很难}
    球队在主场比在客场更容易赢球——在共计40493场比赛中,主队获胜的比率为68.3\%,每场比赛大约有半个进球要归功于主场优势。
  \end{block}
  \pause
  \begin{block}{原因}
    \begin{enumerate}
      \item 去往比赛地点的长途跋涉。
      \item 对场馆的熟悉程度。
      \item 观众的支持。
    \end{enumerate}
  \end{block}
  \pause
  \begin{block}{排除}
    \begin{itemize}
      \item 球队走过路程的长短与在客场输球的趋势毫无关联。
      \item 对场馆的熟悉程度也不太像是主场优势的原因。
    \end{itemize}
  \end{block}
\end{frame}

\begin{frame}
  \frametitle{主场优势 | 猜测}
  \begin{block}{现象}
    \begin{itemize}
      \item 分析了英格兰不同联赛的观众数量,发现观众数量越多,主场优势越大。
      \item 足球裁判只会判罚主队30\%的犯规行为。
      \item 高尔夫球或者网球比赛(裁判的主观判断起到的作用相对较小)就不太存在主场优势。
    \end{itemize}
  \end{block}
  \pause
  \begin{block}{猜测}
    \begin{itemize}
      \item 球迷欢呼的巨浪会使他们的主队获得尤为出色的战绩么?
      \item 有没有可能是数以千计咆哮的球迷让本应公正的裁判变得不公正了呢?
    \end{itemize}
  \end{block}
\end{frame}

\begin{frame}
  \frametitle{主场优势 | 实验}
  \begin{block}{实验}
    \begin{itemize}
      \item 通过屏幕向11名足球运动员、裁判和教练展示52次犯规:26次是客场球队犯下的,26次是主场球队犯下的
      \item 每段犯规录像都会在裁判即将做出正式决定之前暂停,实验参与者必须给出他们的评判
      \item 6位被试裁判看到的是没有声音的场景,5位看到的是有声音的
    \end{itemize}
  \end{block}
  \pause
  \begin{block}{结果与结论:《\textit{The Lancet}柳叶刀》}
    \begin{itemize}
      \item 如果耳中充满球迷的嘈杂叫喊,被试裁判就会做出明显有利于主队的判罚。
      \item 裁判在犹豫不决时,会受到观众的影响。
      \item 半个进球是“场上第12个人”,也就是观看球赛的观众踢进的。
    \end{itemize}
  \end{block}
\end{frame}

\section{电脑}
\begin{frame}
  \frametitle{电脑 | vs. 人}
  \begin{figure}
    \centering
    \includegraphics[width=0.9\textwidth]{c6_ai_01.jpg}
  \end{figure}
\end{frame}

% 2-248,1997 
\begin{frame}
  \frametitle{电脑 | 助人为乐}
  \begin{block}{已知}
    \begin{itemize}
      \item 人类有一种顽固的倾向:为无生命的物质赋予人性。
      \item 在日本,行为规则往往并非只关乎个体,而是涉及整个群组。
    \end{itemize}
  \end{block}
  \pause
  \begin{block}{实验}
    \begin{itemize}
      \item 学生们利用一台电脑解决了一项任务,反过来则要为一台电脑制作一个调色板。
      \item 学生借助一台Windows电脑的帮助完成任务,之后帮助电脑。
    \end{itemize}
  \end{block}
  \pause
  \begin{block}{结果}
    \begin{itemize}
      \item 如果是为此前帮助过他们的那台电脑制作调色板,学生们就会付出更多的时间,几乎达到在另一台同型号电脑上所花费时间的2倍。
      \item 与美国学生不同,日本学生即使没有得到某台电脑的帮助,也会十分乐意帮助它,前提是Windows。
    \end{itemize}
  \end{block}
\end{frame}

% 2-253,1999
\begin{frame}
  \frametitle{电脑 | 礼貌}
  \begin{block}{现象}
    为了不对他人造成伤害,我们容忍些许欺瞒。
  \end{block}
  \pause
  \begin{block}{实验}
    \begin{itemize}
      \item 30名学生,一台教学电脑
      \item 电脑先通过20个问题弄清每个实验参与者的知识水平,然后对此人进行一次与其知识水平相匹配的测试(其实测试都一样)
      \item 学生们对教学电脑的性能做出评价:直接在教学电脑上写 vs. 使用放在其他房间的另一台电脑 vs. 通过纸质问卷写评价
    \end{itemize}
  \end{block}
  \pause
  \begin{block}{结果}
    那些在教学电脑上做出回答的人对其性能的评价明显优于那些用另一台电脑或者纸和笔写评价的人。
  \end{block}
\end{frame}

% 2-257,2000
\begin{frame}
  \frametitle{电脑 | 隐私}
  \begin{block}{已知}
    一个人怎样才能让另一个人在不经意间泄露隐私呢?他只要先泄露关于自己的私人事宜即可。
  \end{block}
  \pause
  \begin{block}{实验}
    \begin{itemize}
      \item 实验参与者必须在电脑上回答11个很私人的问题
      \item 部分参与者遇到特殊情况:每个问题之前都有一段关于电脑信息的说明文字
    \end{itemize}
  \end{block}
  \pause
  \begin{block}{结果}
    被试者在电脑面前严格遵守了人际交往中通用的社交规范:在电脑毫无保留地公布了内心深处的秘密后,使用者们的答案更坦率、更全面、更具体、包含更多细节。
  \end{block}
\end{frame}

\begin{frame}
  \frametitle{电脑 | 启示}
  \begin{block}{启示}
    \begin{itemize}
      \item \alert{所有人与人之间通用的礼节,几乎都在人机关系中得到了遵守},也带来了十分荒谬的结果。
      \item 我们会将一台电脑认知为一个个体,并会回报它对我们的帮助。
      \item 一些复杂的社交礼仪也会延伸到人机接触中。
      \item 为了不伤害一台电脑的感情,我们甚至要对它说谎。
      \item 用户在电脑面前严格遵守人际交往中通用的社交规范。
    \end{itemize}
  \end{block}
\end{frame}

\section{史海撷华}
% 2-23,1874
\begin{frame}
  \frametitle{史海撷华 | 创伤弹道学}
  \begin{block}{疑问}
    枪击究竟为何能够致死?
  \end{block}
  \pause
  \begin{block}{猜测}
    \begin{enumerate}
      \item “碎片”说:极高的温度使子弹熔化,导致许多碎片从子弹上剥落下来。
      \item “离心力”说:旋转的子弹带有离心力,折断了皮肉。
      \item “压力”说:子弹侵入肌肉以及柔软部位后产生了一种压力。
    \end{enumerate}
  \end{block}
  \pause
  \begin{block}{观察}
    \begin{itemize}
      \item 子弹飞出体外的地方、破掉的皮肉并没有扭转出涡旋式的形状——“离心力”的说法并不准确。
      \item 如果子弹没有击中骨骼就不会留下什么碎片——“碎片”的说法也不可靠。
    \end{itemize}
  \end{block}
\end{frame}

\begin{frame}
  \frametitle{史海撷华 | 创伤弹道学}
  \begin{block}{实验}
    \begin{itemize}
      \item 主持:科赫尔(首位获得诺贝尔医学奖的外科医生,1909年)
      \item 武器:口径为10.4、11毫米的“维特尔利”、“夏斯波”
      \item 靶子:杉木板、书、装满沙子的猪膀胱、塞满土豆泥的人类颅骨、用布裹住的尸体、石板、铁皮罐、玻璃板、肝脏……
    \end{itemize}
  \end{block}
  \vspace{-0.3em}
  \pause
  \begin{block}{结果}
     子弹打在空的颅骨上——两个洞 vs. 打在用土豆泥填充的颅骨上——开花。
  \end{block}
  \vspace{-0.3em}
  \pause
  \begin{block}{启示}
    \begin{itemize}
      \item \alert{枪击造成的流体静力冲击才是摧毁人体组织的元凶。}
      \item 为“鲁宾弹”的设计提供理论基础;开创\alert{“创伤弹道学”}研究先河。
      \item \alert{战争的目的}不是尽量消灭更多人的生命,而是“把一个拥有较强战斗能力的敌对者变成一个需要照料的病人”。
    \end{itemize}
  \end{block}
\end{frame}

% 1-193,1966
\begin{frame}
  \frametitle{史海撷华 | 搭车技巧 | 弱不禁风}
  \begin{block}{初衷,《帮助与搭乘》}
    研究汽车司机对待拦车搭乘者的态度。
  \end{block}
  \pause
  \begin{block}{实验}
    男生,用4天时间站在一条4车道的马路边尝试搭车,有时膝盖裹着绷带并拄着拐杖,有时则没有。
  \end{block}
  \pause
  \begin{block}{结果}
    绑绷带拄拐杖时获得了至少双重的搭车机会 $\Longrightarrow$ 搭车技巧之一:要弱不禁风!
  \end{block}
  \pause
  \begin{block}{意义}
    实验结果第一次从科学的角度给搭乘者提供了建议。
  \end{block}
\end{frame}

% 1-231,1971
\begin{frame}
  \frametitle{史海撷华 | 搭车技巧 | 是个女的}
  \begin{block}{《搭乘的成功率》}
    \begin{itemize}
      \item 研究男女在穿着不同服装时,站在路边等待搭乘的成功率。
      \item 研究不同人数和搭配对司机态度的影响。
    \end{itemize}
  \end{block}
  \pause
  \begin{block}{结论}
    \begin{itemize}
      \item 衣着邋遢的比衣着整洁干净的难以搭车。
      \item 男人比女人更难搭车。
      \item 2个男人是最难搭车的。
      \item 单独一个男人搭车的成功率跟一对情侣相差无几。
      \item 成功率最高的是2个女人在一起时。
    \end{itemize}
  \end{block}
\end{frame}

\begin{frame}
  \frametitle{史海撷华 | 搭车技巧 | 更多技巧}
% 1-251,1974
  \begin{block}{目光交流}
    \begin{itemize}
      \item 当被注视时,加州帕罗奥多的600辆车中,有40辆车停了下来。
      \item 如果司机和搭乘者之间没有目光交流,仅有18辆车会停下来。
    \end{itemize}
  \end{block}
  \pause
% 1-252,1975
  \begin{block}{丰胸}
在西雅图进行的一次实验中(寒冷多雨的秋冬季节,所有搭车女子都要在整个过程中穿着滑雪夹克和雨衣),带着垫厚的胸罩(加上5厘米的外延)的女士获得了与不这样做时相比双倍的搭乘机会。 
  \end{block}
\end{frame}

% 1-254,1976 
\begin{frame}
  \frametitle{史海撷华 | 剃须刀教学法}
  \begin{block}{实验室结论}
    \begin{itemize}
      \item “对作为客体的人的稳定的刺激特征”(胡子)会在对人性格的判断上产生深入的影响。
      \item 脸上的一个细小变化,就会让观察者获得不同的印象。
    \end{itemize}
  \end{block}
  \vspace{-0.5em}
  \pause
  \begin{block}{疑问}
      实验室研究中得出的上述结论在现实中是否成立呢?
      %《胡子的魔术——关于长胡子的高校教员对学生所产生的影响的研究》
  \end{block}
  \vspace{-0.5em}
  \pause
  \begin{block}{实验}
    \begin{itemize}
      \item 不留胡子(2学期)——留胡子(1学期)——不留胡子(1学期)
      \item 学期开始见面10分钟后,请学生填写对他个人的评价问卷
    \end{itemize}
  \end{block}
  \vspace{-0.5em}
  \pause
  \begin{block}{结果}
留胡子:不够有目标、不够专心、不够友善、不够坚强、不够理智、不够聪明……;不留胡子:看起来不拘束、比较随意、比较有发展势头……
  \end{block}
\end{frame}

% 2-68,1932 
\begin{frame}
  \frametitle{史海撷华 | 挠痒痒 | 与笑的关联}
  \begin{block}{“笑研究”的漏洞}
    \begin{itemize}
      \item 往往只研究成年人的笑。
      \item 研究渠道不是观察,而是推想和纯粹的理论。
      \item 还很少研究挠痒痒的问题。
    \end{itemize}
  \end{block}
  \pause
  \begin{block}{现象}
小孩子学会在挠痒痒的过程中发笑,是因为挠痒痒一般都发生于玩闹的情境中,而小孩子在玩耍时,本来就是会笑的,笑未必和挠痒痒直接相关。
  \end{block}
  \pause
  \begin{block}{疑问}
    \begin{itemize}
      \item 我们被挠痒痒时为什么会笑?
      \item 挠痒痒和笑之间的关联是天生的吗?
    \end{itemize}
  \end{block}
\end{frame}

\begin{frame}
  \frametitle{史海撷华 | 挠痒痒 | 与笑的关联}
  \begin{block}{实验}
    \begin{itemize}
      \item 克拉伦斯\textbullet 柳巴,自己的第4和第5个孩子
      \item 实验:用纸板面具遮住脸,给孩子挠痒痒
      \item 对照:其余阶段一次都不许给孩子挠痒痒
    \end{itemize}
  \end{block}
  \pause
  \begin{block}{结果与结论}
    \begin{itemize}
      \item 31周大时,第一次在被挠痒痒时自发地笑了起来。
      \item \alert{挠痒痒和笑似乎有着天生的关联。}
    \end{itemize}
  \end{block}
\end{frame}

% 2-168,1970
\begin{frame}
  \frametitle{史海撷华 | 挠痒痒 | 足部挠痒机}
  \begin{block}{疑问}
    人类为何不能对自己挠痒并让自己发笑?
  \end{block}
  \vspace{-0.5em}
  \pause
  \begin{block}{实验}
      足部挠痒机(将挠痒刺激标准化),30名学生,控制方式:自行 vs. 陌生人 vs. 他人控制+被试手握
  \end{block}
  \vspace{-0.5em}
  \pause
  \begin{block}{结果}
    \begin{itemize}
      \item 与自行控制相比,一个陌生人控制操纵杆会让他们明显感到刺痒。
      \item 自己手握操纵杆但由他人控制时,发痒敏感度虽然有所减弱,却仍然比自己控制时要强。
    \end{itemize}
  \end{block}
  \vspace{-0.5em}
  \pause
  \begin{block}{结论——《\textit{Nature}自然》}
    \begin{itemize}
      \item 得知被挠痒的时间和位置,并不足以完全压制发痒感,除非本人掌握控制权。想要压制发痒感,被挠痒者本人必须掌握控制权。
      \item “人类会什么会怕痒”这一根本问题仍然是个谜。
    \end{itemize}
  \end{block}
\end{frame}

% 2-231,1994
\begin{frame}
  \frametitle{史海撷华 | 挠痒痒 | 机器挠痒痒}
  \begin{block}{疑问}
    机器能挠痒痒吗?如果挠痒痒方式完全相同,只是一个来自人类,一个来自机器,是否会造成什么区别?挠痒痒具有社会性功能?
  \end{block}
  \pause
  \begin{block}{实验}
    \begin{itemize}
      \item 一台挠痒痒机器人(完全不起作用),被试(塞住耳朵、蒙上眼睛)
      \item 被试被挠2次痒:1次由人来挠,1次由机器来挠(事实上2次都是由一人完成的)
      \item 比较被试者在被人以及被机器挠痒痒时发笑的强度(面部视频录像——被试自我评价)
    \end{itemize}
  \end{block}
  \pause
  \begin{block}{结果与结论}
    \begin{itemize}
      \item 发笑的强度始终相同,与被人还是被“机器”挠痒痒并无关联。
      \item 挠痒痒时的发笑并没有任何社会性的特征,只是一种反射行为。
    \end{itemize}
  \end{block}
\end{frame}

% 2-22,1822
\begin{frame}
  \frametitle{史海撷华 | 圣经 | 鬣狗}
  \begin{block}{缘起}
采石场的工人们发现了一个洞穴,里面堆满了老虎、牡鹿、熊、马、大象、犀牛、河马和鬣狗的骨头。
  \end{block}
  \vspace{-0.5em}
  \pause
  \begin{block}{解释与疑问}
    \begin{itemize}
      \item 解释:教士借助《圣经》故事——野兽们被“大洪水”冲进洞穴。
      \item 疑问:洞里没有沙石的踪迹。大型动物是怎样通过狭窄的洞口挤入洞中的呢?
    \end{itemize}
  \end{block}
  \vspace{-0.5em}
  \pause
  \begin{block}{猜测与实验}
    \begin{itemize}
      \item 观察:骨头上有被啃噬的痕迹,而且牙印与鬣狗牙齿完全吻合。
      \item 推断:鬣狗将觅得的腐尸拖进洞里慢慢享用。
      \item 实验:从非洲南部引进鬣狗,让它啃食骨头。
      \item 结果:不吃的部位都能在洞穴里找到,吃掉的部位正是洞穴里没有的。实验用的骨头和洞穴里的骨头的断裂状况完全一样。
    \end{itemize}
  \end{block}
\end{frame}

% 2-70,1932
% 2-186,1984
\begin{frame}
  \frametitle{史海撷华 | 圣经 | 十字架刑}
  \begin{block}{疑问}
    \begin{itemize}
      \item 人的身体不可能只靠三只钉子就挂住?
      \item 钉在十字架上死去的生理过程。
    \end{itemize}
  \end{block}
  \pause
  \begin{block}{实验}
    \begin{itemize}
      \item 1个十字架、1把锤头、3只钉子、1具尸体、12只“刚切下来的手臂”、若干只脚
      \item 一座2.3米高的(带有腕带的)十字架
    \end{itemize}
  \end{block}
  \pause
  \begin{block}{结果}
    \begin{itemize}
      \item 被钉上十字架的耶稣最终死于窒息?
      \item 耶稣死于心脏停跳和呼吸中断,是由大量失血和创伤性休克所致。
    \end{itemize}
  \end{block}
\end{frame}

% 1-292,1998
\begin{frame}
  \frametitle{史海撷华 | 圣经 | 耶利哥的扬声器}
  \begin{block}{耶利哥的扬声器}
    \begin{itemize}
      \item 7个祭祀在约柜前吹响他们的羊角,使得耶利哥的城墙倾陷——《圣经\textbullet 约书亚记》
      \item 美国教育节目《学习频道》,加利福尼亚的怀勒实验室,怀勒WAS 3000扬声器(相当于10000个扬声喇叭)
      \item 结果:经过6分钟的噪声作用,砂浆真的开始破碎,小墙分崩离析
      \item 结论:(众所周知的事实)声音真的可以产生破坏
      \item 事实:耶利哥所属的迦南城根本没有设防……
    \end{itemize}
  \end{block}
\end{frame}

% 2-114,1960
\begin{frame}
  \frametitle{史海撷华 | 眼睛}
  \begin{block}{“心灵之窗”}
    \begin{itemize}
      \item 大脑从事某些特定活动时,瞳孔大小会发生怎样的变化?
      \item 被观看者的瞳孔大小会对观看者产生什么影响?
    \end{itemize}
  \end{block}
  \pause
  \begin{block}{灵感}
    某天晚上,妻子观察到他在翻看一本动物画册时,瞳孔突然放大。
  \end{block}
  \pause
  \begin{block}{实验}
    \begin{itemize}
      \item 助理,“招贴画女郎”实验(在一大叠风景图片中混入一张“香艳热辣”的美女照片)——瞳孔直径测量
      \item 2男2女,不同图片(婴儿、裸体男/女性、残疾儿童、现代艺术……)
      \item 男士,一位女性的2张照片(唯一区别在于瞳孔大小)
    \end{itemize}
  \end{block}
\end{frame}

\begin{frame}
  \frametitle{史海撷华 | 眼睛}
  \begin{figure}
    \centering
    \includegraphics[width=0.9\textwidth]{c6_eye_01.png}
  \end{figure}
\end{frame}

\begin{frame}
  \frametitle{史海撷华 | 眼睛}
  \begin{block}{结果}
    \begin{itemize}
      \item 图片内容为婴儿、母亲怀抱婴儿、裸体男性时,女性被试者的瞳孔明显放大;男性被试者对裸体女性的反应尤为强烈。
      \item 男性被试者观看瞳孔放大的那张照片时,自己的瞳孔也放大得特别明显。
      \item 不管观看内容是正面还是负面的,只要观看者产生了关注的兴趣,瞳孔就会放大;大脑在努力加工信息时,瞳孔也会放大。
      \item 检测人类思想活动的终极手段(性取向,预测不同产品的市场价值)?
    \end{itemize}
  \end{block}
  \pause
  \begin{block}{未解之谜}
    \begin{itemize}
      \item 瞳孔为什么会随着大脑活动而运动呢?
      \item 这种现象蕴含着什么深刻的意义吗?
      \item 它只是大脑在处理其他事情的时候产生的副产品吗?
    \end{itemize}
  \end{block}
\end{frame}

% 2-125,1962
\begin{frame}
  \frametitle{史海撷华 | 极度恐惧}
  \begin{block}{疑问}
    极度恐惧如何影响脑力活动效率?
  \end{block}
  \vspace{-0.3em}
  \pause
  \begin{block}{实验}
    \begin{itemize}
      \item 20个新兵,阅读紧急情况指示,飞机迫降,填写调查问卷
      \item 新兵所在位置误划为炮火攻击目标,炮击/放射性污染/森林大火,维修无线电设备
    \end{itemize}
  \end{block}
  \vspace{-0.3em}
  \pause
  \begin{block}{结果}
    \begin{itemize}
      \item 20人中只有5人没有在“紧急情况”下惊慌失措。其他人都“对死亡或受伤产生了不同程度的恐惧”。
      \item 填写问卷的过程中,惶恐情绪体现得最为明显,尤其是在回忆安全指令时,很多人忘记了近乎一半的内容。
      \item 只有炮击场景会分散新兵修理无线电设备的注意力,另外2种场景(放射性污染和森林大火)并没有影响他们的效率。
    \end{itemize}
  \end{block}
\end{frame}

% 2-135,1964
\begin{frame}
  \frametitle{史海撷华 | 厌恶疗法}
  \begin{block}{问题}
    如何消除酒瘾?如何对付过度贪玩、暴食症或性取向异常?
  \end{block}
  \pause
  \begin{block}{方法}
在酒徒的杯子里放几只蜘蛛(不愉快的刺激),注射药物导致呼吸停止(产生极大的恐惧),实施电击、让病人闻刺鼻的气味或者服用引起呕吐的药物。
  \end{block}
  \pause
  \begin{block}{厌恶疗法}
    将欲戒除的行为和不愉快的刺激进行结合。
  \end{block}
  \pause
  \begin{block}{启示——标准负面案例}
    以实验为目的,让人毫不知情地陷入极度恐惧,是一件不可容忍的事情。
  \end{block}
\end{frame}

% 2-141,1965
\begin{frame}
  \frametitle{史海撷华 | 加芬克尔化}
  \begin{block}{缘起}
    \begin{itemize}
      \item 人们在说话时,信息表达往往很不完整。令人吃惊的是,它没有给任何人带来困扰。精确细致的表达或者持续不断的追问反而会让人不胜其烦。
      \item \alert{无障碍的交流大多是建立在语言含混的基础上的}——这听起来像是一个悖论。
    \end{itemize}
  \end{block}
  \pause
  \begin{block}{启示}
    \begin{itemize}
      \item 人们从含义不明的句子中整合出固定的意义,加芬克尔将这种策略称为“俗民方法学”。
      \item \alert{我们的交流非常依赖共同的背景知识和隐含的猜测。}——“破坏性实验”
      \item 有些美国人会将有意打破默认的文化规则的行为称为“加芬克尔化”。
    \end{itemize}
  \end{block}
\end{frame}

% 2-273,2003
\begin{frame}
  \frametitle{史海撷华 | 音乐}
  \begin{block}{疑问}
      一种普遍的人类行为方式得以产生,往往是因为:从长远来看,这种行为方式一定能够带来更多后代。音乐究竟与之有何联系,则完全是个谜。动物是否也喜欢音乐?
  \end{block}
  \vspace{-0.5em}
  \pause
  \begin{block}{实验}
    6只狨猴,V形笼子:一侧动听和弦,一侧刺耳可怕的音调组合。
  \end{block}
  \vspace{-0.5em}
  \pause
  \begin{block}{结果}
    它们聆听和谐与不和谐音调组合的时间一样长;猴子对刮擦黑板声和另外一个音量相等的声响之间没有表现出对任何一方的偏好,它真正喜欢的东西——安静。
  \end{block}
  \vspace{-0.5em}
  \pause
  \begin{block}{启示}
    对音乐的适应可能是人类独有的一种特征;\alert{音乐是个极具人类特性的事件}。
  \end{block}
\end{frame}

\begin{frame}
  \frametitle{史海撷华 | 感受幽默}
  \begin{block}{初衷}
    \begin{itemize}
      \item 一些大脑右侧受伤的人会变得很没有幽默感,他们知道笑话的意思,但并不觉得滑稽可笑。
      \item 大脑是如何理解双关语的?
    \end{itemize}
  \end{block}
  \vspace{-0.5em}
  \pause
  \begin{block}{实验}
    \begin{enumerate}
      \item 在被试者的左侧或右侧视野(分别对应右脑和左脑)展示一个与双关语有关的词
      \item 分析每种情况下被试者的反应时间,确定哪一侧大脑在其中起主要作用
    \end{enumerate}
  \end{block}
  \vspace{-0.5em}
  \pause
  \begin{block}{\alert{结果}}
左侧大脑是语言半球,负责处理双关语的本意,稍后右侧半球才会领会到这个词的言外之意。(人们首先用左半球结合上下文做出特定的解释,右半球随后引导人们发现另一层意想不到的意义,幽默就出现了。)
  \end{block}
\end{frame}

\begin{frame}
  \frametitle{史海撷华 | 其他}
  \begin{block}{真正的疯狂}
  \begin{itemize}
    % 2-262,2001
    \item 勇于说不——不要低估了激情对自身行为的影响
    % 2-267,2002
    \item 面部识别——整体 vs. 单个特征,脸 vs. 模拟肖像
  \end{itemize}
  \vspace{-1em}
  \begin{figure}
    \centering
    \includegraphics[width=0.7\textwidth]{c6_face_01.png}
  \end{figure}
  \end{block}
\end{frame}


\input{snippet/class_tail}
