\input{snippet/beamer_head}
\input{snippet/class_head}

%\includeonlyframes{current}

\title[生长发育]{第三章\quad 疯狂的实验——生长发育}
\author[Yixf]{伊现富(Yi Xianfu)}
\institute[TIJMU]{天津医科大学(TIJMU)\\ 生物医学工程与技术学院}
\date{2019年3月}

\input{snippet/beamer_toc.tex}

% 2-50,1926
\section{幼儿喂养}
\begin{frame}
  \frametitle{幼儿喂养 | 问题}
  \begin{block}{旧式的育儿观念}
    从母乳改为成人食品的调整过程需要持续3~4年。
  \end{block}
  \pause
  \begin{block}{营养学}
    动物(包括人)能够在丰富的食物供应中凭借直觉选择出最适合自身成长发展的食物,还是必须遵从生物化学家制定的食谱,先由他们分析每种食物的营养成分,然后再做选择?
  \end{block}
\end{frame}

\begin{frame}
  \frametitle{幼儿喂养 | 实验}
  \begin{block}{实验}
    \begin{itemize}
      \item 实验对象:15名6个月到4岁半不等的婴幼儿
      \item 实验食品:(从30多种花样的食品中挑出10种食物和2种饮料)苹果、菠萝泥、西红柿、煎土豆、煮熟的小麦、玉米、燕麦和黑麦、做好并剁碎的牛肉、骨髓、动物的脑、肝、肾、剁碎的鱼肉、鸡蛋、食盐、水、不同种类的牛奶和橙汁,……
      \item 实验控制:不接受父母的干预,没有儿科医生的指令
      \item 实验测量:称量吃掉的分量,检查生长发育状况
    \end{itemize}
  \end{block}
\end{frame}

\begin{frame}
  \frametitle{幼儿喂养 | 结果}
  \vspace{-0.5em}
  \begin{block}{结果(37500人次)}
    \begin{itemize}
      \item 孩子们的生长发育完全正常,没有任何营养缺乏的迹象。
      \item 每个儿童的首选“菜单”不仅差异明显,还会固定偏爱某种食品。
      \item 孩子们吃水果、肉类、蛋类、脂肪类食物的数量远远超过了当时儿科医生的建议量,吃粮食和蔬菜的数量则少于建议量。
      \item 菠菜几乎遭到所有孩子的厌弃,圆白菜和生菜也不太受欢迎。
      \item 乍一看来,孩子们的营养搭配乱七八糟;仔细观察,他们都做出了明智合理的营养选择:蛋白质、脂肪和碳水化合物的数量都在标准范围之内。
    \end{itemize}
  \end{block}
  \vspace{-0.5em}
  \pause
  \begin{block}{结论}
    \begin{itemize}
      \item 所有人都遵循的标准食谱未必就是最优的膳食搭配。
      \item 儿童可以毫无障碍地消化成人食品并正常地长大。
      \item 儿童具有不可思议的直觉能力,可以从任何备选食物中搭配出均衡的饮食\alert{!?}(未加工、未调味也未放糖)
    \end{itemize}
  \end{block}
\end{frame}

% 1-157,1958
\section{母爱}
\begin{frame}
  \frametitle{母爱 | 依恋 | 缘起}
  \begin{block}{当时的观点}
    \begin{itemize}
      \item 乳儿实现最佳成长的首要条件是充足的食物和干净的环境。
      \item 依恋母亲是种次要情感,它在母亲解决子代更加重要的需求(比如饥饿和干渴)的前提下才会产生。
    \end{itemize}
  \end{block}
  \pause
  \begin{block}{培养教学实验所用恒猴的过程中偶然发现}
    为了避免猴子生病,将出生不久的小猴与母猴分离:
    \begin{itemize}
      \item 猴子吃住无忧,更壮更重
      \item 总是蜷坐在笼中吮吸手指、呆望远处
      \item 全然不知如何与众多小公/母猴相处
      \item 对铺在笼中的布制衬垫表现出强烈的依恋
      \item 抱着衬垫、用衬垫包裹自己,更换衬垫时会大叫
      \item 布料被夺去的前5天,它们几乎难以度日
    \end{itemize}
  \end{block}
\end{frame}

\begin{frame}
  \frametitle{母爱 | 依恋 | 实验}
  \begin{block}{猜测}
    该现象会不会是在提示柔软的布同瓶子里的奶一样重要?
  \end{block}
  \vspace{-0.5em}
  \pause
  \begin{block}{“母亲机”}
      “绒布妈妈” vs. 挂着奶瓶的“铁丝妈妈”;阴险残忍的“怪物妈妈”
  \end{block}
  \vspace{-0.5em}
  \pause
  \begin{block}{结果}
    \begin{itemize}
      \item 小猴子们每天紧紧拥抱“绒布妈妈”超过12小时,只有当它们感到饥渴时,才会爬到“铁丝妈妈”身上一会儿。
      \item 只要“怪物妈妈”刚一恢复平静,它们就重新跑回紧紧依偎在妈妈身边。
    \end{itemize}
    \vspace{-2.2em}
    \begin{figure}
      \centering
      \visible<3->{\includegraphics[width=0.25\textwidth]{c3_mother_01.jpg}}\qquad
      \visible<3->{\includegraphics[width=0.2\textwidth]{c3_mother_02.jpg}}
    \end{figure}
  \end{block}
\end{frame}

\begin{frame}
  \frametitle{母爱 | 依恋 | 结论}
  \begin{block}{结论}
    \begin{itemize}
      \item 乳儿对母亲依恋的首要原因是母亲躯体的柔软温暖,与该躯体是否同时作为食物来源并无关系。
      \item 身体上的亲近对于孩童的成长非常重要。
      \item 幼童追寻母亲并且完全依赖母亲。
    \end{itemize}
  \end{block}
  \pause
  \begin{block}{启示}
    \begin{itemize}
      \item 多陪陪孩子(但不要溺爱孩子)
      \item 老人如小孩——多陪陪父母
    \end{itemize}
  \end{block}
\end{frame}

% 2-79,1935
\begin{frame}
  \frametitle{母爱 | 智商}
  \begin{figure}
    \centering
    \includegraphics[width=0.45\textwidth]{c3_iq_01.jpg}
    \includegraphics[width=0.52\textwidth]{c3_iq_02.jpg}
  \end{figure}
  \pause
  \begin{quote}
    “过去,我坐在角落里,整天只做一件事情,就是晃动上身,一直晃到这两位开始进行实验。今晚,我能来到这里,是因为他们给我带来了爱和理解。”
  \end{quote}
   \qquad ——布兰卡(斯基尔斯13个孩子中的一个)在智力障碍研究奖(斯基尔斯,斯柯达克)上的颁奖发言
\end{frame}

\begin{frame}
  \frametitle{母爱 | 智商 | 实验}
  \begin{block}{当时的观点}
    \begin{itemize}
      \item 人的智商主要取决于遗传,一生之中基本不变。
      \item 只要满足基本的生理需求,儿童就可以健康成长。
      \item 幼年时期获得过多的温情和关爱是有害的。
    \end{itemize}
  \end{block}
  \vspace{-0.6em}
  \pause
  \begin{block}{巧合——“大胆试验”——结局——巧合}
    \begin{itemize}
      \item 心理学家哈罗德\textbullet 斯基尔斯
      \item 13名3岁以下的儿童:和智障女性(“妈妈”)待在一起
      \item 12名儿童(比对样本):留在福利院
    \end{itemize}
  \end{block}
  \vspace{-0.6em}
  \pause
  \begin{block}{结果}
    \begin{itemize}
      \item 儿童期的智商测试:平均提高28分(进步最大的往往有一个固定的“妈妈”) vs. 下降26分
      \item 成年后的生活状态:11人结婚、完整的人生 vs. 9人未婚、1人离异、生活在社会边缘
    \end{itemize}
  \end{block}
\end{frame}

\begin{frame}
  \frametitle{母爱 | 启示}
  \begin{block}{福利院(/敬老院)}
    \begin{itemize}
      \item 福利院里有很多看起来发育迟缓、精神涣散的孩子,其实他们并没有先天缺陷,只是他们得到的刺激和关注太少了。
      \item \alert{孤儿福利院原来是滋生智力障碍的温床!}
    \end{itemize}
  \end{block}
  \pause
  \begin{block}{智商}
    \begin{itemize}
      \item 关爱和温情让她们走出了迟钝和麻木的世界。
      \item 智力高低并非生来注定,还会受到环境影响,幼年环境尤为重要。
      \item 在智商问题大论战中,“遗传说”和“教育说”仍在激烈交锋。
    \end{itemize}
  \end{block}
\end{frame}

% 1-206,1969
\section{自我意识}
\begin{frame}
  \frametitle{自我意识}
  \begin{block}{问题}
    动物是否具有自我意识?——《动物有意识吗?》
  \begin{figure}
    \centering
    \includegraphics[width=0.9\textwidth]{c3_self_01.jpg}
  \end{figure}
  \end{block}
\end{frame}

\begin{frame}
  \frametitle{自我意识 | 实验}
  \begin{block}{实验:刮胡子的灵感——简单而天才的方法}
表现自我感知最简单的工具是镜子:
    \begin{itemize}
      \item 实验组:照镜子——涂红点——照镜子
      \item 对照组:涂红点——照镜子
    \end{itemize}
  \end{block}
  \pause
  \begin{block}{结果}
    \begin{itemize}
      \item 在有镜子的情形下,黑猩猩触碰红点的次数比没有镜子时多25次。
      \item 从未照过镜子的黑猩猩对着镜子也不会触摸眉毛和耳朵上的红点。
    \end{itemize}
  \end{block}
\end{frame}

\begin{frame}
  \frametitle{自我意识 | 实验}
  \begin{block}{引申}
    许多动物行为学家在自己孩子身上做的第一项科学实验!
    \begin{figure}
      \centering
      \includegraphics[width=0.5\textwidth]{c3_self_02.jpg}\quad
      \includegraphics[width=0.44\textwidth]{c3_self_03.jpg}
    \end{figure}
  \end{block}
\end{frame}

\begin{frame}
  \frametitle{自我意识 | 启示}
  \begin{block}{本质}
    \begin{itemize}
      \item 动物拥有辨认镜中自我的能力,到底意味着什么?
      \item 是否意味着它已经可以明确认识到那是自己?
      \item 是否意味着它也能同样辨别出同类?
      \item 是否意味着它也会像人类那般做出计划或者撒谎?
    \end{itemize}
  \end{block}
  \pause
  \begin{block}{启示}
    \begin{itemize}
      \item 在科学史上偶然间冒出绝妙灵感的事例层出不穷。
      \item 许多科学实验能够带给人有趣的答案,然而却未必真正触及了问题的本质。
    \end{itemize}
  \end{block}
\end{frame}

% 2-73,1933
\section{思维发展}
\begin{frame}
  \frametitle{思维发展 | 现象}
  \begin{block}{现象}
    \begin{itemize}
      \item 数量守恒——某个东西被分成若干部分或者改变了形状,其总量并不会突然增多或者减少
      \item 客体永恒性——事物具有恒定不变的特征
      \item 空间参考系统——涉及对“水平”和“垂直”的认识,反映出空间观念的发展和对空间能力的运用
    \end{itemize}
  \end{block}
  \pause
  \begin{block}{实验}
    \begin{itemize}
      \item 让\textbullet 皮亚杰,不同年龄段的儿童
      \item 数量守恒实验
      \item 水平面任务
    \end{itemize}
  \end{block}
\end{frame}

\begin{frame}
  \frametitle{思维发展 | 启示}
  \begin{block}{儿童思维发展理论}
    \begin{itemize}
      \item 儿童思维的发展过程要经过几个前后相继的“建设阶段”。
      \item 儿童要在特定的年龄才能达到某个阶段,从儿童所犯的典型错误能够识别出他们正处于什么阶段。
      \item 在前运算阶段(2-7岁),儿童的判断主要由感知决定。(理解不了某些过程是可逆的)
      \item 在具体运算阶段(7-12)岁,儿童开始按照逻辑规则去思考。(数量守恒,同时注意多个元素)
    \end{itemize}
  \end{block}
  \pause
  \begin{block}{启示}
    \begin{itemize}
      \item 儿童的思维错误就是宝贵的知识源泉。
      \item 皮亚杰的实验绝对堪称心理学界最重要和最富创造力的实验。
      \item 皮亚杰虽然是极为杰出的思想者,却不是特别严谨的实验员。
      \item 皮亚杰通过水平面任务测试说明了儿童空间想象力的发展。
    \end{itemize}
  \end{block}
\end{frame}

% 2-84,1936
% 2-201,1991
\begin{frame}
  \frametitle{思维发展 | 水平面任务}
  \begin{figure}
    \centering
    \includegraphics[width=0.9\textwidth]{c3_water_01.png}
  \end{figure}
\end{frame}

\begin{frame}
  \frametitle{思维发展 | 水平面任务}
  \begin{block}{儿童}
    \begin{itemize}
      \item 5岁以下:画在瓶子中央,像一团乱麻(没有“水面”的概念)。
      \item 年龄稍长:画出的水平面总是与杯壁垂直。
      \item 6或7岁:发现垂直是不对的,但水平面还是被画成了倾斜状。
      \item 7、8岁左右:已经开始向正确答案靠拢。
      \item 大约9岁:发现正确的答案——水平面总是呈现水平状态,也就是与桌面平行。
    \end{itemize}
  \end{block}
  \pause
  \begin{block}{成人}
    \begin{itemize}
      \item 女性在水平面任务中的表现明显比男性差。(为什么?)
      \item 错误率随着经验的增加而提高?(经验诱导人们将玻璃杯作为参照系?)
      \item 会写中文的人在水平面任务中得分更高。
    \end{itemize}
  \end{block}
\end{frame}

% 2-109,1958
\begin{frame}
  \frametitle{思维发展 | 幼儿思维能力}
  \begin{block}{问题}
    人类的哪些技能是生来具备的,哪些技能是后天习得的,先天基础和外部环境,哪个起着决定作用?
  \end{block}
  \pause
  \begin{block}{当时的观点}
    新生儿就像一张干净的白纸,世界在他们眼中只是一些或明或暗的色块,随着“观看”经验与日俱增,他们才逐渐学会分辨不同的事物。
  \end{block}
  \pause
  \begin{block}{实验}
    \begin{itemize}
      \item 缘起:刚刚破壳而出的小鸡会有选择地啄食谷粒大小的球体(似乎天生就有辨别事物的能力)
      \item 实验:罗伯特\textbullet 法恩茨,22/30个婴儿(1-15周不等),观看各种形状,观察条纹板和纯色板,球体 vs. 圆环,观察加法题
    \end{itemize}
  \end{block}
\end{frame}

\begin{frame}
  \frametitle{思维发展 | 幼儿思维能力}
  \begin{block}{结果}
    \begin{itemize}
      \item 都喜欢观察比较复杂的图形——婴儿天生就具备区分这些图形的能力。
      \item 确定婴儿的视力——一个月大的婴儿能够识别3mm宽的条纹,半岁大的可以识别再细10倍的条纹。
      \item 婴儿生来就能看出立体的东西吗?——看球体的时间比看圆环的时间更长。
      \item 婴儿能认出人脸吗?——人脸更受欢迎。
      \item 婴儿能够运算。
    \end{itemize}
  \end{block}
\end{frame}

% 2-154,1968
\section{延迟满足}
\begin{frame}
  \frametitle{延迟满足 | 实验}
  \begin{block}{预测}
    如何预测一名4岁儿童的未来——他能否成为一个健全而知足的人(优异成绩、很多朋友、吸毒、和谐伴侣……)?
  \end{block}
  \pause
  \begin{block}{棉花糖测试}
    让他选择是现在吃到1块棉花糖,还是以后吃到2块。——孩子愿意为2块棉花糖等待的时间越长,他在以后的人生中就表现得越好。
    \vspace{-0.8em}
    \begin{figure}
      \centering
      \includegraphics[width=0.39\textwidth]{c3_marshmellow_01.jpg}\ 
      \includegraphics[width=0.58\textwidth]{c3_marshmellow_02.jpg}
    \end{figure}
  \end{block}
\end{frame}

\begin{frame}
  \frametitle{延迟满足 | 启示}
  \begin{block}{延迟满足}
    应该马上服从自己的需求,还是为了一个更高的目标推迟这些需求呢?
  \end{block}
  \vspace{-0.5em}
  \pause
  \begin{block}{结果}
    \begin{itemize}
      \item 成长过程中没有\alert{父亲陪伴}的儿童往往不想等待更大的奖励。
      \item 在棉花糖测试里显示出耐心的儿童,在学校的表现明显好于其他人,也较少出现各种问题。
      \item 4-6岁完成的棉花糖测试以出人意料的准确度预测了被试儿童10年后的许多特征。
    \end{itemize}
  \end{block}
  \vspace{-0.5em}
  \pause
  \begin{block}{启示}
    \begin{itemize}
      \item 自发地延迟满足是迈向成熟的重要一步。
      \item \alert{“为长期目标放弃短期诱惑”是人生中最应该掌握的重要能力之一。}
      \item 告诫:实验结果深受实验组织方式和奖励类型的影响,而且归根结底也只反映出一种统计学上的倾向,就个体案例来说指导作用有限。
    \end{itemize}
  \end{block}
\end{frame}

\begin{frame}
  \frametitle{延迟满足 | 未解之谜}
  \begin{block}{未解之谜}
    \begin{itemize}
      \item 我们能不能训练这种能力?
      \item 如何可以的话,应该怎样训练呢?
      \item 训练真的能够对以后的人生产生正面影响吗?
      \item 延迟满足的能力有可能是基因决定的吗?
    \end{itemize}
  \end{block}
  \begin{figure}
    \centering
    \includegraphics[width=0.6\textwidth]{c3_marshmellow_06.jpg}\quad
    \includegraphics[width=0.35\textwidth]{c3_marshmellow_07.jpg}
  \end{figure}
\end{frame}

\begin{frame}
  \frametitle{延迟满足 | “棉花糖测试”新解}
  \begin{block}{2018年5月,《心理科学》}
    \begin{itemize}
      \item 最初的研究样本量不到90,而且都是斯坦福大学附近幼儿园的孩子(家庭经济状况本身就不同凡响)。
      \item 测试了900多个孩子,来自各种社会背景,父母的种族、受教育水平和收入有较大差别(900人中有500个孩子的母亲没有接受过高等教育)。
      \item 能够预测一个孩子延迟满足水平的最佳指标,不是别的,恰恰是他们的家庭收入(\alert{来自富裕家庭的孩子有更好的意志力})。
      % \item 那些能忍受长时间棉花糖诱惑的孩子,只是因为幸运地出生在富裕家庭里而已。
      \item 最终能更好地预测15岁成绩的指标,是家庭收入水平以及教育环境。
      \item 对于贫困家庭的孩子来说,日常生活充满着不确定性,忍耐并不一定能带来长远的收益。对于他们来说,今朝有酒今朝醉的策略,实际上比延迟满足更优。
      % \item 棉花糖测试本身只能说明孩子对于现状的不确定性的看法:如果眼前的财富可能分分钟消失,未来的横财会不会到来根本不重要。
      \item 拨开意志力的遮羞布,棉花糖测试反映的不过是在经济这只无形的手的操控下,年幼的孩子对于当下的成见而已。
    \end{itemize}
  \end{block}
\end{frame}

\begin{frame}
  \frametitle{延迟满足 | “棉花糖测试”新解}
  \begin{block}{贫穷的影响——2013年,《科学》}
    \begin{itemize}
      \item 印度泰米尔纳德邦以种植甘蔗为生的农民在甘蔗收割前(没钱)的智商比收割后(有钱)的智商要低。
      \item 处于贫困中的人的智商可以下降13分(人群平均智商为100,大多数人在70-130分之间)。
      \item 贫困的一大特征就是生活中充满了不可预测性,贫困迫使人活在当下,因为朝不保夕时,思考未来简直就是一种奢侈。
      \item 活在当下的策略也会损害智力,这或许就是家庭收入影响儿童未来发展的一种方式。
      \item 贫困不但影响智力,贫富差距也会影响父母养育孩子的方式;\alert{贫困本身会改变人类的行为和看问题的方式}。
      % \item 和富裕家庭相比,贫困家庭的父母更容易对孩子索要糖果的请求妥协。
      % \item 贫困的父母在力所能及的时候更愿意满足孩子的愿望,而更富裕家庭的父母却可能要求孩子等待更大的蛋糕。
    \end{itemize}
  \end{block}
\end{frame}

% 1-49,1900
\section{效果定律}
\begin{frame}
  \frametitle{效果定律 | 实验}
    \vspace{-0.5em}
  \begin{block}{目的}
    找寻一种办法来研究老鼠的智慧:
    \begin{itemize}
      \item 研究需要在一个可控的环境中进行
      \item 老鼠的行为要尽可能不受到非自然环境的干扰
    \end{itemize}
  \end{block}
    \vspace{-0.5em}
  \pause
  \begin{block}{灵感}
    老鼠有走弯路的喜好
  \end{block}
    \vspace{-0.5em}
  \pause
  \begin{block}{设备}
    建一座迷宫(2.4 $\times$ 1.8米):地面上铺撒锯屑,迷宫中心放置食物
    \vspace{-0.8em}
    \begin{figure}
      \centering
      \includegraphics[width=0.45\textwidth]{c3_mouse_01.jpg}\quad
      \includegraphics[width=0.3\textwidth]{c3_mouse_04.jpg}
    \end{figure}
  \end{block}
\end{frame}

\begin{frame}
  \frametitle{效果定律 | 结果与启示}
  \begin{block}{结果}
    \begin{itemize}
      \item 前两只老鼠因为受到实验室中噪声的惊吓,没能完成任务
      \item 第三次实验中一只雄性老鼠用时15m走到迷宫中央,第四次实验用时10m,第五次1m45s,第六次3m,第七次50s
    \end{itemize}
  \end{block}
  \pause
  \begin{block}{效果定律}
    能引导生物达到满意效果(找到食物)的行为(穿过迷宫的正确道路),较之带来不好结果的行为,更可能被加强和重现。
  \end{block}
\end{frame}

% 1-97,1930
\section{斯金纳箱}
\begin{frame}
  \frametitle{斯金纳箱 | 简介}
  \begin{block}{目的}
    \begin{itemize}
      \item 找寻一种仪器用来测量老鼠的行为
      \item 希望获悉新的行为是如何产生的(而不是简单地研究既得的反射)
    \end{itemize}
  \end{block}
  \pause
  \begin{block}{斯金纳箱}
    一个研究动物行为的自动化装置。
    \vspace{-0.8em}
  \begin{figure}
    \centering
    \includegraphics[width=0.34\textwidth]{c2_skinnerbox_01.jpg}\quad
    \includegraphics[width=0.62\textwidth]{c2_skinnerbox_02.jpg}
  \end{figure}
  \end{block}
\end{frame}

\begin{frame}
  \frametitle{斯金纳箱 | 结论}
  \begin{block}{结论}
    \begin{itemize}
      \item 斯金纳用一种简单的方法发现了老鼠的行为改变的一个方面:一种行为发生的频率的改变(压杠杆的间隔越来越短)。
      \item 动物在这一过程中不是靠先天的反应,而是靠学会了一种新的行为方式。
    \end{itemize}
  \end{block}
  \pause
  \begin{block}{“动作条件反射”理论(区别于巴甫洛夫的经典条件理论)}
    \begin{enumerate}
      \item 生物体持续呈现本能行为
      \item 肯定或否定的结果增加或减少了生物体重复这一行为的可能性
      \item 环境决定上述结果
    \end{enumerate}
  \end{block}
\end{frame}

\begin{frame}
  \frametitle{斯金纳箱 | 应用}
  \begin{block}{行为主义研究}
    专门描述外界刺激对人和动物行为的影响:
    \begin{itemize}
      \item 当老鼠需要连续5次压杠杆才能获得食物,或是在偶然的压杆次数之后获得食物,会有怎样的行为呢?
      \item 当它在某个行为实施后遭到惩罚,结果又会怎样?
      \item 如何消除一个习得的行为?
      \item 利用动物的感官训练动物完成各种不同的任务
      \item ……
    \end{itemize}
  \end{block}
\end{frame}

\begin{frame}
  \frametitle{斯金纳箱 | 启示}
  \begin{block}{启示}
    \begin{itemize}
      \item \alert{奖励强化一种(好的)行为,惩罚弱化一种(坏的)行为。}
      \item \alert{不要等到动物达到总的目标时才给予奖励,而是对每一个小的进步都要奖励。}
      \item 世界就是一只巨大的斯金纳箱,人类所有的行为理论都可以由此找出答案。
      \item 教育学应用:有赏有罚、赏罚分明;多鼓励、勤鼓励……
    \end{itemize}
  \end{block}
\end{frame}

\section{史海撷华}
% 1-73,1914
\begin{frame}
  \frametitle{史海撷华 | 通往香蕉的阶梯}
  \begin{block}{人}
    “把第二个建筑材料放到第一个上只是对把第一个建筑材料放到地上这一行为的重复”。
  \end{block}
  \pause
  \begin{block}{黑猩猩}
    黑猩猩需要解决的问题“分为两个截然不同的部分:一部分它可以轻易解决,而另一部分解决时却遇到了相当大的困难”。容易解决的部分是把一只箱子推到香蕉下面,困难的部分是把第二只箱子叠垛到先前的那只上面。
  \end{block}
  \pause
  \begin{block}{启示}
    人和黑猩猩相似的行为并不能证明它们相似的思维方式。
  \end{block}
\end{frame}

% 1-103,1931
\begin{frame}
  \frametitle{史海撷华 | 猴妹妹}
  \begin{block}{经验/现象}
    \begin{itemize}
      \item 从小看大,三岁看老;三岁定八十
      \item 狼/熊/虎/猴孩;《狼的孩子雨与雪》
    \end{itemize}
    \vspace{-1em}
    \begin{figure}
      \centering
      \includegraphics[width=0.5\textwidth]{c3_wolf_01.jpg}
    \end{figure}
  \end{block}
  \pause
  \begin{block}{假说/猜想}
    \begin{itemize}
      \item 想要把生命早期已形成的习性再次革除并非易事?
      \item 在成长过程中,是本性还是文化,是环境还是遗传,起了决定性作用?
    \end{itemize}
  \end{block}
\end{frame}

\begin{frame}
  \frametitle{史海撷华 | 猴妹妹}
  \begin{block}{实验}
    \begin{itemize}
      \item 不可行:把一个智商正常的婴孩儿扔到荒郊野外,研究其行为。
      \item 反其道而行之:10个月大的唐纳德,7个月大的古亚(雌黑猩猩);猴子和人类的孩子被用完全相同的方式对待9个月
    \end{itemize}
  \end{block}
  \vspace{-0.5em}
  \pause
  \begin{block}{猜测}
    如果猴子没有像孩子一样成长,那就是动物遗传的本性占了支配地位;倘若猴子表现出了典型的孩童行为,那就证明环境的力量发挥了主导作用。
  \end{block}
  \vspace{-0.5em}
  \pause
  \begin{block}{结果/后果}
    \begin{itemize}
      \item 古亚表现出对人类环境惊人的适应性;唐纳德是一个更优秀的模仿者。
      \item 本想把一只猴子教育成人,却把一个人教成了猴子。
      \item 唐纳德,一名精神病医生,(45岁)自杀
    \end{itemize}
  \end{block}
\end{frame}

% 2-60,1932
\begin{frame}
  \frametitle{史海撷华 | 先天 vs. 教育}
  \begin{block}{实验人员}
    \begin{itemize}
      \item 实验者:莫特尔\textbullet 麦格劳(发现“潜游反射”)
      \item 实验对象:约翰尼、吉米(同卵双胞胎?异卵双胞胎!)
    \end{itemize}
  \end{block}
  \vspace{-0.5em}
  \pause
  \begin{block}{问题}
    \begin{itemize}
      \item “促进措施”对儿童运动技能发展的影响
      \item 先天因素和教育因素哪个作用更大
    \end{itemize}
  \end{block}
  \vspace{-0.5em}
  \pause
  \begin{block}{成果:约翰尼}
    \begin{itemize}
      \item 13个月,会溜旱冰
      \item 15个月,完成从1.5米高的跳板上俯身跃下的动作
      \item 17个月,在水下潜游4米
      \item 21个月,能从1.6米高的台子上爬下来
      \item 22个月,毫不费力地攀爬上70度的陡峭斜坡
    \end{itemize}
  \end{block}
\end{frame}

% 2-211,1992
\begin{frame}
  \frametitle{史海撷华 | 玩具偏好}
  \begin{block}{现象}
    儿童做事通常只有3分钟热度,可是,不同性别的儿童对特定玩具的偏好却稳定而持久。
  \end{block}
  \vspace{-0.5em}
  \pause
  \begin{block}{实验——10年后结果发表}
    \begin{itemize}
      \item 88只绿长尾猴——44只雌性和44只雄性
      \item 6种玩具:2种典型的男性玩具(球和警车),2种典型的女性玩具(布娃娃和厨房用锅),2种中性玩具(图画书和毛绒玩具狗)
      \item 结果:雄性猴子玩球和警车的时间是雌性的2倍;雌性猴子玩布娃娃和厨房用锅的时间是雄性的2倍;图画书和毛绒玩具狗在雌性和雄性猴子中的受欢迎程度基本相当。
    \end{itemize}
  \end{block}
  \vspace{-0.5em}
  \pause
  \begin{block}{结论}
    不同性别的个体对不同玩具的偏好并不仅仅由家长和电视广告决定,生物学也在发挥作用。(这种不同的偏好从何而来?)
  \end{block}
\end{frame}

\begin{frame}
  \frametitle{史海撷华 | 社群智力}
  \vspace{-0.5em}
  \begin{block}{“社群智力假说”}
    复杂的社群行为使得动物拥有了复杂的认知能力。
  \end{block}
  \vspace{-0.5em}
  \pause
  \begin{block}{木箱实验}
    悬吊放有一大块生牛肉的木箱,要想吃到肉,需要拉动木箱上垂下的绳子,从而打开弹簧门闩。
  \end{block}
  \vspace{-0.5em}
  \pause
  \begin{block}{结果}
    \begin{itemize}
      \item 狮子
        \begin{itemize}
          \item 11/12只成功吃到了牛肉(7只独立找到方法,4只通过观察学会)
          \item 5至7个月后,11只狮子中仍有10只还记得木箱的正确打开方式
        \end{itemize}
      \item 狮子的表现优于美洲豹和老虎
    \end{itemize}
  \end{block}
  \vspace{-0.5em}
  \pause
  \begin{block}{启示}
    \begin{itemize}
      \item 老大憨,老二奸,调皮捣蛋属老三。
      \item 打开家门走向社会;读万卷书行万里路;知识、见识、胆识;……
    \end{itemize}
  \end{block}
\end{frame}

\begin{frame}
  \frametitle{生长发育 | 总结}
  \begin{block}{养育孩子的“正确”观点}
    \begin{itemize}
      \item 客观评价孩子的“挑食”习惯与个人偏好。
      \item 物质基础重要,心灵关爱更更更重要。
      \item 母亲的关爱重要,父爱的陪伴同样重要。
      \item 思维发展具有阶段性,揠苗助长只会适得其反。
      \item 多鼓励、勤鼓励,奖赏与惩罚并重。
      \item 早期习惯的养成和戒除至关重要。
      \item 多带孩子走出家门、融入同龄集体。
      \item 事有利弊,需要恰当的权衡进行抉择。
    \end{itemize}
  \end{block}
\end{frame}


\input{snippet/class_tail}
