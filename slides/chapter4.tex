\input{snippet/beamer_head}
\input{snippet/class_head}

%\includeonlyframes{current}

\title[生物学]{第四章\quad 疯狂的实验——生物学}
\author[Yixf]{伊现富(Yi Xianfu)}
\institute[TIJMU]{天津医科大学(TIJMU)\\ 生物医学工程与技术学院}
\date{2017年3月}

\input{snippet/beamer_toc.tex}

% 
\section{}

% 
\section{}

% 
\section{}

% 
\section{}

% 
\section{}


\section{史海撷华}
% 1-6,1620
\begin{frame}
  \frametitle{史海撷华 | 光合作用}
  \begin{block}{种柳行动}
    \begin{itemize}
      \item 范\textbullet 黑尔蒙特(最后一个炼金术士,第一个化学家),首位用泥土、树木和称实际操作实验的人
      \item 200磅在炉中烘干的泥,5磅重的柳树幼枝,定期浇水
      \item 5年后拔出柳树,对土和柳树分别称重
    \end{itemize}
  \end{block}
  \vspace{-0.75em}
  \pause
  \begin{block}{结果与结论}
    \begin{itemize}
      \item 泥土减重2盎司,树木增重164磅零3盎司
      \item 164磅的木质、树皮以及根系都只来源于水(当时唯一合理的结论)
    \end{itemize}
  \end{block}
  \vspace{-0.75em}
  \pause
  \begin{block}{启示}
    \begin{itemize}
      \item 为实验铺平了道路,使其从此成为获取认识的手段
      \item 他的想法启发了很多的学者,开展罐中植物的研究
      \item 为“光合作用”这一神秘过程的探究开了先河
      \item 学生借此测试洞察力,练习严谨的实验设计
    \end{itemize}
  \end{block}
\end{frame}

% 1-8,1729
\begin{frame}
  \frametitle{史海撷华 | 生物钟 | 植物}
  \begin{block}{现象}
    \begin{itemize}
      \item 含羞草在夜间合拢叶片,白天打开。
      \item 如果把含羞草置于一个它无法知晓昼夜的环境中,情况会怎样呢?
    \end{itemize}
  \end{block}
  \pause
  \begin{block}{实验}
    \begin{itemize}
      \item 1729年,让\textbullet 雅克\textbullet 徳奥图斯\textbullet 德迈朗(天文学家)
      \item 把一株含羞草放到漆黑的柜中
    \end{itemize}
  \end{block}
  \pause
  \begin{block}{结果}
    \begin{itemize}
      \item 叶片在没有太阳光的情况下,还可以有规律地开合。
    \end{itemize}
  \end{block}
  \pause
  \begin{block}{启示}
    \begin{itemize}
      \item “时间生物学”(研究生物体内部的生物钟)的创立者
    \end{itemize}
  \end{block}
\end{frame}

% 1-109,1938
\begin{frame}
  \frametitle{史海撷华 | 生物钟 | 人}
  \begin{block}{疑问}
    人的睡眠规律究竟只是习惯,抑或是人体内存在着生物钟?
  \end{block}
  \pause
  \begin{block}{实验}
    \begin{itemize}
      \item 实验一:把生物钟从每天24小时调整为48小时、12小时——实验无果而终
      \item 实验二:把生物钟从每天24小时调整为21小时、28小时,测量体温——结果模棱两可
      \item 实验三:宽20米、高8米的猛犸洞窟(漆黑、安静、恒温)——结果显示出两面性
    \end{itemize}
  \end{block}
  \pause
  \begin{block}{启示}
    随后的实验证实,人体内确实存在着生物钟。它的运转大致跟一天24小时相吻合,并且每天都会根据实际时长进行自动调整。
  \end{block}
\end{frame}

% 1-12,1774
\begin{frame}
  \frametitle{史海撷华 | 温度耐受}
  \begin{block}{疑问}
    人体能够承受什么样的温度?
  \end{block}
  \pause
  \begin{block}{实验}
    \begin{itemize}
      \item 建造桑拿室,45$^{\circ}$C-100$^{\circ}$C-127$^{\circ}$C
      \item 穿衣服发汗,赤裸着手持一只平底锅,上面放着一块牛排
    \end{itemize}
  \end{block}
  \pause
  \begin{block}{结果及结论}
    \begin{itemize}
      \item 24页的《皇家协会学报》
      \item 有一个“与生命体直接相关的自然的系统”消除热量?(降温——汗液等的蒸发加之以血液流动!)
    \end{itemize}
  \end{block}
\end{frame}

% 1-16,1802
\begin{frame}
  \frametitle{史海撷华 | 蛙腿实验}
  \begin{block}{现象}
    如果用两种不同的金属触碰蛙腿,连成通路,它们的肌肉会抽搐。
  \end{block}
  \pause
  \begin{block}{理论}
    \begin{itemize}
      \item 加尔瓦尼:“动物电流”与生命力有关,效果与电流通过无生命的物质是不同的。
      \item 伏特:世界上只有一种电,如论市雷雨天的闪电还是抽搐的蛙腿,原理都与这种电有关。
    \end{itemize}
  \end{block}
  \pause
  \begin{block}{启示}
    \begin{itemize}
      \item 《弗兰肯斯坦》(第一部科幻小说)——玛丽\textbullet 雪莱
      \item 眨眼睛的尸体——对于绞死者进行头颅实验
    \end{itemize}
  \end{block}
\end{frame}

% 1-20,1825
\begin{frame}
  \frametitle{史海撷华 | 胃上有洞的人}
  \begin{block}{实验材料}
       1882年,威廉\textbullet 博蒙特;亚力克西斯\textbullet 圣马丁(受伤的士兵)
  \end{block}
  \vspace{-0.5em}
  \pause
  \begin{block}{疑问}
消化仅仅是一个纯化学的过程,还是同时需要人体提供某种未知的生命力量促使其完成?消化和腐烂的区别是否在于前者拥有人体内的未知生命力量,而后者没有?
  \end{block}
  \vspace{-0.5em}
  \pause
  \begin{block}{实验}
      用丝线系好的食物,塞入胃中——拉出,观察消化情况;插入软管、导出胃液,把一把牛肉粒置于其中;……
  \end{block}
  \vspace{-0.5em}
  \pause
  \begin{block}{结果}
    \begin{itemize}
      \item 器皿中胃液的化学能量足以完成消化(无需未知的生命力量)。
      \item 推翻了胃液仅仅是流于胃中储存起来的唾液的推测。
      \item 医学伦理
    \end{itemize}
  \end{block}
\end{frame}

% 
\begin{frame}
  \frametitle{史海撷华 | }
\end{frame}

% 
\begin{frame}
  \frametitle{史海撷华 | }
\end{frame}

% 
\begin{frame}
  \frametitle{史海撷华 | }
\end{frame}

% 
\begin{frame}
  \frametitle{史海撷华 | }
\end{frame}

% 
\begin{frame}
  \frametitle{史海撷华 | 其他}
  \begin{block}{真正的疯狂}
    \begin{itemize}
      \item 蚯蚓没有听觉——达尔文为蚯蚓演奏巴松管、笛子和钢琴
      \item 
      \item 
      \item 
      \item 
      \item 
      \item 
      \item 
      \item 
    \end{itemize}
  \end{block}
\end{frame}


\input{snippet/class_tail}
